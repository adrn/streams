\documentclass[preprint]{aastex}

\begin{document}

Galaxies, by mass, are mostly composed of \emph{Dark Matter}
\citep{}. Simulations of structure and galaxy formation -- both with
and without baryons -- lead us to believe that such \emph{halos} of
dark matter have universal density profiles \citep{} and are triaxial
in shape \citep{}. Including baryons in such simulations tends to
soften the triaxiality and alters the radial density profile. Thus,
precise measurements of the shape, orientation, and radial profile of
dark matter halos provides information about both dark matter physics
and the baryonic processes that helped shape them. Tidal streams from
infalling satellite galaxies have been used to constrain properties of
the Milky Way's halo \citep{}; typically found outwards of 20~kpc, the
orbits of associated stars are still largely under the influence of
the dark matter and thus serve as potentiometers \citep{}. Using a new
method that utilizes full phase-space (6D) kinematic information for a
small sample ($\sim$100s) of stars stripped from a known progenitor
(e.g., Sagittarius; Johnston\&Price-Whelan 2013, in prep), it is
possible to accurately constrain the axis ratios, total mass, and
orientation of the Milky Way's dark matter halo. We plan to use a
sample of RR Lyrae stars selected from catalogs of RRab stars from
time-domain, photometric surveys such as LINEAR \citep{}, PTF \citep{}
and CSS \citep{}. This method relies on a technique for deriving
distances to individual RR Lyrae stars with 2\% accuracy \citep{},
along with anticipated GAIA proper motion errors of
$\sim10-20~\mathrm{mas}/\mathrm{yr}$ for such stars at $\sim$
20-30~kpc. This leaves one remaining velocity component -- the radial
velocity -- which we propose to measure using the 2.4m Hiltner
telescope at MDM, equipped with the [Allyson ??].

\end{document}
