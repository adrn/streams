\documentclass[preprint]{aastex}

\begin{document}
\title{Probing the Milky Way's Dark Matter Halo with RR Lyraes in
  Tidal Streams I: The Sagittarius Leading Arm} \author{Adrian M.
  Price-Whelan\altaffilmark{1}, Kathryn Johnston\altaffilmark{1},
  Allyson Sheffield\altaffilmark{1}} \altaffiltext{1}{Department of
  Astronomy, Columbia University, Mail Code 5246, New York, NY 10027}

Galaxies, by mass, are mostly composed of \emph{Dark
  Matter}. Simulations of structure and galaxy formation -- both with
and without baryons -- lead us to believe that such \emph{halos} of
dark matter have universal density profiles \citep{nfw96} and are
triaxial in shape \citep{jing02}. Including baryons in such
simulations tends to soften the triaxiality and alters the radial
density profile. Thus, precise measurements of the shape, orientation,
and radial profile of dark matter halos provides information about
both dark matter physics and the baryonic processes that helped shape
them. Tidal streams from infalling satellite galaxies have been used
to constrain properties of the Milky Way's halo \citep[e.g.][]{law10};
typically found outwards of 20~kpc, the orbits of associated stars are
still largely under the influence of the dark matter and thus serve as
potentiometers \citep[e.g.][]{johnston99}. Using an updated method
that utilizes full phase-space (6D) kinematic information for a small
sample ($\sim$100s) of stars stripped from a known progenitor (e.g.,
Sagittarius; Price-Whelan et al 2013, in prep), it is possible to
accurately constrain the axis ratios, total mass, and orientation of
the Milky Way's dark matter halo. This method relies on a technique
for deriving distances to individual RR Lyrae stars with 2\% accuracy
\citep{madore12}, along with anticipated GAIA proper motion errors of
$\sim10-20~\mu\mathrm{as}/\mathrm{yr}$ for such stars at $\sim$
20-30~kpc. This leaves one remaining velocity component -- the radial
velocity.  We plan to observe a sample of RR Lyrae stars selected to
be likley members of Sgr's debris system using catalogs of RRab stars
from time-domain, photometric surveys such as LINEAR \citep{linear}, PTF
\citep{ptf} and CSS \citep{catalina}.  Our observations will both confirm
membership of the debris system in order to construct a target list
for Spitzer (proposals are due in August 2013) as well as measure the
missing velocity dimension.

We propose to measure radial velocities of RR Lyraes in the Sgr
leading arm using ModSpec on the 2.4m Hiltner telescope at MDM.  Using
the 600 groove mm$^{-1}$ grating blazed at 5000 \AA\ and a 2048x2048
detector with 15 micron pixels (Wilbur), we will have wavelength
coverage of about 1000 \AA\, including the H$\beta$ line at 4861
\AA\ and the [OI] sky line at 5577 \AA\ (the latter to aid in the
radial velocity determination by identifying any systematic offsets).
We will collect bias and flat field frames at the beginning and end of
the night, and ThAr arc lamp spectra for wavelength calibration will
be taken throughout the night to account for telescope flexure.
Radial velocity standards will be selected that are of a similar
spectral type as the targets (e.g., HD122693 -- a dwarf of spectral
type F8).  Typical equatorial coordinates for the targets are
($\alpha,\delta$)=(13$^{h}-16^{h}$,-5$^{\circ}- +25^{\circ}$); stars
in this range will be at airmass $<2$ all night.  The typical $R$-band
magnitude is 16.  To achieve S/N of 30 per pixel with the proposed
set-up, an integration time of 1200 seconds is needed for a target
with $R$=16.  Considering the total amount of observing time of 9
hours per night and the overhead needed for calibration frames, we
anticipate observing approximately 40 RR Lyraes.  Reductions will be
carried out by Sheffield using a well-developed reduction pipeline for
processing ModSpec data.

\begin{thebibliography}{ }
\bibitem[Jing \& Suto(2002)]{jing02} Jing, Y.~P., \& Suto, Y.\ 2002, \apj, 574, 538
\bibitem[Navarro et al.(1996)]{nfw96} Navarro, J.~F., Frenk, C.~S., \& White, S.~D.~M.\ 1996, \apj, 462, 563
\bibitem[Law \& Majewski(2010)]{law10} Law, D.~R., \& Majewski, S.~R.\ 2010, \apj, 714, 229 
\bibitem[Johnston et al.(1999)]{johnston99} Johnston, K.~V., Zhao, H., Spergel, D.~N., \& Hernquist, L.\ 1999, \apjl, 512, L109 
\bibitem[Madore \& Freedman(2012)]{madore12} Madore, B.~F., \& Freedman, W.~L.\ 2012, \apj, 744, 132 
\bibitem[Stokes et al.(2000)]{linear} Stokes, G.~H., Evans, J.~B., Viggh, H.~E.~M., Shelly, F.~C., \& Pearce, E.~C.\ 2000, \icarus, 148, 21 
\bibitem[Law et al.(2009)]{ptf} Law, N.~M., Kulkarni, S.~R., Dekany, R.~G., et al.\ 2009, \pasp, 121, 1395 
\bibitem[Drake et al.(2013)]{catalina} Drake, A.~J., Catelan, M., Djorgovski, S.~G., et al.\ 2013, \apj, 763, 32 
\end{thebibliography}

\end{document}
