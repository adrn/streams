% MDMPROP.TEX -- MDM electronic proposal form.
% Revision: September 4, 2003 --- valid for the Fall 2010 observing semester.
% DEADLINE: 5:00 PM Friday, April 16, 2010
%
% General Instructions:
%
%   I. Where/how/when to submit this form:
%
%      Give a paper copy to Jules or his mailbox on the 10th floor
%      of Pupin by the above deadline, or email the postscript or
%      pdf file to jules@astro.columbia.edu.
%
% THE FORM STARTS HERE

% please don't modify or delete this line.
\documentstyle[mdmprop,11pt]{article}

\begin{document}

% Give a descriptive title for the proposal in the \title command.
%
%    \title{TEXT}
%
% Note that a title can be quite long; LaTeX will break the title into
% separate lines automatically.  If you wish to indicate line breaks
% yourself, do so with a `\\' command at the appropriate point in
% the title text.

\title{Probing the Milky Way's Dark Matter Halo with RR Lyraes in Tidal Streams I: The Triangulum-Andromeda Stellar Feature}

% Investigator's (PI and CoI) information blocks.
%
% The PI must be affiliated with Columbia University.
% Please give each investigator's name, email address.
% At least one investigator must be listed as definitely
% being present at the telescope.
%
%    \name{OBSERVER NAME}
%    \emailaddress{EMAIL ADDRESS}
%    \atthetelescope{Y/N}
%
% DO NOT remove the \begin{PI} and \end{PI} or the \begin{CoI} and
% \end{CoI} lines.

\begin{PI}
\name{Adrian Price-Whelan}			% REQUIRED
\emailaddress{adrn@astro.columbia.edu}		% REQUIRED
\atthetelescope{Y}       % REQUIRED
\end{PI}

\begin{CoI}
\name{Allyson Sheffield}			% REQUIRED
\emailaddress{asheffield@astro.columbia.edu}		% REQUIRED
\atthetelescope{Y}       % REQUIRED
\end{CoI}

\begin{CoI}
\name{Kathryn Johnston}			% REQUIRED
\emailaddress{kvj@astro.columbia.edu}		% REQUIRED
\atthetelescope{N}       % REQUIRED
\end{CoI}

\begin{CoI}
\name{Branimir Sesar}			% REQUIRED
\emailaddress{bsesar@astro.caltech.edu}		% REQUIRED
\atthetelescope{N}       % REQUIRED
\end{CoI}


% You can supply more CoI blocks, but only the first three will be
% printed on the form.

% Give a general abstract of the scientific justification appropriate for
% a non-specialist.  Write the abstract between the \begin{abstract} and
% \end{abstract} lines.  Limit yourself to approximately 175 words.

% DO NOT remove the \begin{abstract} and \end{abstract} lines.

\begin{abstract}
This proposal requests time to obtain radial velocity measurements for a sample of RR Lyrae in the Triangulum-Andromeda stellar cloud -- one of several new, amorphous stellar structures discovered recently in the Galactic halo. Models suggest that, like the more well-known stellar streams, these clouds could result from the disruption of satellites during the hierarchical formation of our Galaxy. Thus, this debris is equally suited for probing the shape, orientation, and radial profile of dark matter in the Galactic halo. At the distance of TriAnd ($\sim$ 25~kpc), we will soon have accurate proper motions from GAIA. Combined with accurate distances using the mid-IR period-luminosity relation for RR Lyrae (TODO: something about the Spitzer proposal?), we only need radial velocity measurements to obtain full 6-D phase-space information for this sample of stars. We will then use this 6D information to model the orbit of the progenitor and subsequently infer new constraints on the mass, axis ratios, and orientation of the Milky Way's dark matter halo.

\end{abstract}

% Indicate whether this proposal is part of your PhD thesis work by
% putting a Y or an N inside the \thesis{} curly braces.

\thesis{N}

% Indicate whether you are requesting long-term status by putting
% a Y or an N inside the \longterm{} curly braces.  If you do want
% long-term status (a "Y" answer), please tell us the number of nights,
% number of semesters, and what telescopes by filling in \longtermdetails{}.
% Please be brief; \longtermdetails is limited to one line.

\longterm{}
\longtermdetails{}

% List the details of the observing runs being requested, for
% UP TO THREE runs.  The parameters for each run are segregated
% between \begin{obsrun} and \end{obsrun} lines.  Please be sure
% that the information is isolated properly for each run.
%
%   \telescope{}	For example, \telescope{1.3m}
%   \instrument{}	For example, \instrument{Mk III + Charlotte}
%   \lunardays{}	For example, \lunardays{14}
%   \optimaldates{}	For example, \optimaldates{Sep 1 - Nov 30}
%   \acceptabledates{}	For example, \acceptabledates{Sep 1 - Dec 15}
%
%
% Instrument combinations may be specified with "+".
%
% \numnights should give the number of nights of the run
% \lunardays should specify the maximum number of nights from new moon
% which can be utilized to accomplish your scientific goals.  It should
% be a number from 1 (new moon) to 14 (full moon).
%
% \optimaldates should contain the range of OPTIMAL dates.
%
% \acceptabledates should give the range of ACCEPTABLE dates (i.e., you
% would not accept time outside those limits).
%
% To enter the acceptable and optimal date ranges, please use two
% dash-separated dates with 3-letter abbreviations for the month
% (Jan, Feb, Mar, Apr, May, Jun, Jul, Aug, Sep, Oct, Nov, Dec)
% followed by the day.  For example:  \optimaldates{Feb 15 - Apr 23}
%
% If you need to enter two or more ranges of acceptable or optimal dates
% for a single observing run, separate the ranges by commas.  For example:
% \acceptabledates{Oct 7 - Nov 24, Jan 7 - Jan 31}
% It is not necessary to give date ranges based on lunar phase information.
% For instance, if you wish to observe an object in March through May
% within five days of new moon, you may give \lunardays{5} and
% \optimaldates{Mar 1 - May 31} instead of multiple shorter date ranges.
%
% DO NOT remove the \begin{obsrun} and \end{obsrun} blocks.

\begin{obsrun}
\telescope{2.4-m}
\instrument{ModSpec, Wilbur, 600 l mm$^{-1}$ blazed at 5000 \AA}
\numnights{5}
\lunardays{grey}
\optimaldates{Oct 4 - Oct 8}
\acceptabledates{Oct 1 - Oct 20}
\end{obsrun}

\begin{obsrun}
\telescope{}
\instrument{}
\numnights{}
\lunardays{}
\optimaldates{}
\acceptabledates{}
\end{obsrun}

\begin{obsrun}
\telescope{}
\instrument{}
\numnights{}
\lunardays{}
\optimaldates{}
\acceptabledates{}
\end{obsrun}

% You may NOT supply more obsrun blocks.  Three is the limit.

% If there are dates that you cannot use for non-astronomical reasons,
% (i.e., other than moon phase or when your object is up)
% please give the dates by filling in the curly braces in \unusabledates{}.
% Please be brief; \unusabledates is LIMITED TO ONE LINE.

\unusabledates{}

% In the following "essay question" sections, the delimiting pieces of
% markup (\justification, \feasibility, etc.) act as LaTeX \section*{}
% commands.  If the author wanted to have numbered subsections within
% any of these, LaTeX's \subsection could be used.

% SCIENTIFIC JUSTIFICATION
%
% Give the scientific justification for the proposed observations.
% This section should consist of paragraphs of text (and may include
% EPS figures) that follow the \justification line.
% Try to include an explanation of the overall significance to astronomy.

% In order to include an EPS plot, you should use the LaTeX "figure"
% environment.  The plot file is included with the \plotone{FILENAME}
% command; two side-by-side plot files can be included by typing
% \plottwo{FILENAME1}{FILENAME2}.  Use \caption{} to specify a caption.
% The \epsscale{} command can be used to scale \plotone plots if they
% appear too large on the printed page.
%
% \begin{figure}
% \epsscale{0.85}
% \plotone{sample.eps}
% \caption{Sample figure showing important results.}
% \end{figure}
%
% If you need to rotate or make other transformations to a figure, you may
% use the \plotfiddle command:
% \plotfiddle{PSFILE}{VSIZE}{ROTANG}{HSCALE}{VSCALE}{HTRANS}{VTRANS}
% \plotfiddle{sample.eps}{2.6in}{-90.}{32.}{32.}{-250}{225}
% where HSCALE and VSCALE are percentages and HTRANS and VTRANS are
% in PostScript units, 72 PS units = 1 inch.
%
% If you wish to use the "reference" environment, follow
% the following example:
%
%\begin{references}
%\reference Armandroff \& Massey 1991 AJ 102, 927.
%\reference Berkhuijsen \& Humphreys 1989 A\&A 214, 68.
%\reference Massey 1993 in Massive Stars: Their Lives in the Interstellar
%  Medium (Review), ed. J. P. Cassinelli and E. B. Churchwell, p. 168.
%\reference Massey, Armandroff, \& Pyke 1993, in prep.
%\end{references}

\justification Galaxies, by mass, are mostly composed of \emph{Dark
  Matter}. Simulations of structure and galaxy formation -- both with
and without baryons -- lead us to believe that such \emph{halos} of
dark matter have universal density profiles \citep{nfw96} and are
triaxial in shape \citep{jing02}. Including baryons in such
simulations tends to soften the triaxiality and alters the radial
density profile. Thus, precise measurements of the shape, orientation,
and radial profile of dark matter halos provides information about
both dark matter physics and the baryonic processes that helped shape
them. Tidal streams from infalling satellite galaxies have been used
to constrain properties of the Milky Way's halo \citep[e.g.][]{law10};
typically found outwards of 20~kpc, the orbits of associated stars are
still largely under the influence of the dark matter and thus serve as
potentiometers \citep[e.g.][]{johnston99}. Using an updated method
that utilizes full phase-space (6D) kinematic information for a small
sample ($\sim$100s) of stars stripped from a known progenitor (e.g.,
Sagittarius; Price-Whelan et al 2013, in prep), it is possible to
accurately constrain the axis ratios, total mass, and orientation of
the Milky Way's dark matter halo. This method relies on a technique
for deriving distances to individual RR Lyrae stars with 2\% accuracy
\citep{madore12}, along with anticipated GAIA proper motion errors of
$\sim10-20~\mu\mathrm{as}/\mathrm{yr}$ for such stars at $\sim$
20-30~kpc. This leaves one remaining velocity component -- the radial
velocity.

We plan to observe a sample of RR Lyrae stars selected to be likley
members of the TriAnd structure using a catalogs of RRab stars from
the Palomar Transient Factory \citep[PTF][]{ptf}

%%AAS insert some background on TriAnd here, including recent (unpublished) results from KVJ models showing that the feature is part of a stream rather than a cloud

\begin{references}


\end{references}

%\begin{figure}
%\epsscale{1.00}
%\plotone{HA_hist.eps}
%\caption{Counts of 2MASS M giants. The left-hand panel shows M giant star counts in the latitude range $30^{\circ} < |b| < 50^{\circ}$ for the northern Galactic hemisphere at negative l (dotted bold line) and southern Galactic hemisphere at positive l (solid bold line). Gray lines indicate counts in the corresponding north/south positive/negative l regions. The right hand panels show the distance distribution for M giants in the bold histogram (south) in the left hand panel in the longitude ranges $15^{\circ} < l < 40^{\circ}$ (black) and $0^{\circ} < l < 15^{\circ}$ (gray).  The magnitude range for stars in the figure is $9.5 \le K_{S,0} \le 12.5$}
%\end{figure}


% FEASIBILITY
%
% Assess the technical and scientific feasibility of the observations.
% This section should consist of text and tables only (no figures)
% following the \feasibility line.
%
% List objects, coordinates, and magnitudes (or surface brightness,
% if appropriate), desired S/N, wavelength coverage and resolution.
% Justify the number of nights requested as well as the specific
% telescope(s), instruments, and lunar phase.  Indicate the optimal
% detector, as well as acceptable alternates.  If you've requested
% long-term status, explain why this is necessary for successful
% completion of the science.

\feasibility
Our program aims to map the velocity profile of the southern portion of the Hercules-Aquila (HerAq) stellar
cloud.
The RR Lyraes that we will target spans the sky in the region roughly $0^{h}<\alpha<4^{h}$ and $30^{\circ}<\delta<40^{\circ}$.
This range of equatorial coordinates is best observed in October 2013.

All targets are selected as RR Lyraes.
The range in magnitude is $16.2 \le R \le 18.2$.

Using the 600 groove mm$^{-1}$ grating blazed at 5000 \AA\ and a 2048x2048 detector with 15 micron pixels (Wilbur), we will have wavelength coverage of about 1300 \AA\, including the H$\beta$ line at 4861 \AA\ and the [OI] sky line at 5577 \AA\ (the latter to aid in the radial velocity determination by identifying any systematic offsets).
This set-up will give an expected resolution of 3-4 \AA.
We will collect bias and flat field frames at the beginning and end of the night, and ThAr arc lamp spectra for wavelength calibration will be taken throughout the night to account for telescope flexure.  Radial velocity standards will be selected that are of a similar spectral type as the targets (e.g., XXXX  -- a dwarf of spectral type F8).


We expect to
be able to get an RV precision of 5 km s$^{-1}$;
We prefer S/N $>30$ to get good metallicity estimates as well.
Using the Exposure Time Calculator we expect to be able to
achieve S/N=30 per pixel in XXX sec for
a $R=16$ RR Lyrae, and in XXX sec for a ....
To determine the wavelength solution, comparison lamp spectra will be taken throughout the night, specifically after a long telescope slew. In addition, quartz lamp spectra and bias frames will be obtained at the beginning of the night.
With 1-2 hours every night reserved for obtaining
RV standards and calibration frames, we can expect to observe about
10 targets per night.
We request 5 nights to obtain spectra of several dozen potential HerAq cloud members.

Tech Spec Summary:

Groove density: 600 lines mm$^{-1}$

Blaze $\lambda$=5000 \AA

CCD=``Wilbur'': 15 $\mu$m pixels

% OTHER FACILITIES
%
% Why MDM? If you are using other facilities for this project,
% explain how the MDM observations fit into the scheme of things.
%
% This section should consist of text and tables only (no figures)
% following the \feasibility line.

\whymdm
ModSpec on the 2.4-m Hiltner Telescope is a propitious choice for our HerAq cloud radial velocity survey -- the spectral band covered will provide not only precision RVs but also an estimate of the metallicities.
% PAST USE
%
% List your allocation of telescope time at MDM during the
% past 3 years, and describe the status of the project (cite
% publications where appropriate).  Mark any allocations of time
% related to the current proposal with a \relatedwork{} command.
% Are your MDM observations achieving their goals?

\thepast
I used Mark III on the 1.3-m telescope in January.
The run was very successful --- I obtained spectra for close to 200 M giants, and the RVs are preliminarily at the 10-20 km s$^{-1}$ precision level.

I used ModSpec on the 2.4-m telescope in May.
This was an excellent run.  The SkyCalc application made the acquisition of guide stars a very simple task (more time was needed to do this on the January 1.3-m run).

I have an upcoming run on the 2.4-m in November, which is aimed at targeting the TriAnd star cloud.



\end{document}
