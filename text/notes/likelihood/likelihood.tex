\documentclass[letterpaper,12pt,preprint]{aastex}

% packages
\usepackage{amssymb,amsmath, amsbsy}

% commands
\newcommand{\given}{\,|\,}
\newcommand{\dd}{\mathrm{d}}
\newcommand{\transpose}[1]{{#1}^{\mathsf{T}}}
\newcommand{\inverse}[1]{{#1}^{-1}}
\newcommand{\jac}[1]{\left\vert \J{#1} \right\vert}

% bold X and D
\newcommand{\D}{{\bf D}}
\newcommand{\W}{{\bf W}}
\newcommand{\X}{{\bf X}}
\newcommand{\J}{{\boldsymbol J}}
\newcommand{\bSigma}{{\bf \Sigma}}
\newcommand{\bsigma}{\boldsymbol\sigma}
\newcommand{\rtide}{R_{\rm tide}}

\begin{document}

\title{Back-integration likelihood}
\author{Adrian M. Price-Whelan}

Take $\D$ to be the observed, heliocentric 6D position of a star --- e.g., the measured position on the sky, ($l$, $b$); distance, $D$; proper motions, ($\mu_l$, $\mu_b$); and line-of-sight velocity, $v_r$ --- and take $\W$ to be the corresponding \emph{true}, 6D position of the same star in the same coordinate and reference frame. Assume that we have determined that this star was once part of a progenitor system (e.g., satellite galaxy) tidally disrupting and forming a cold debris structure in the potential, $\Phi$, of some parent galaxy. The position of the satellite is observed to be at heliocentric 6D position $\D_s$, with true position $\W_s$, where the subscript $s$ refers to the satellite. The true, 6D spherical, heliocentric positions of the star and satellite -- $\W$ and $\W_s$ -- have corresponding 6D cartesian, Galactocentric coordinate vectors $\X$ and $\X_s$. Initially, we assume the present-day mass of the satellite, $M_s$, and velocity dispersion, $\sigma_v$, have been measured perfectly (but we will return to this later). With this notation, we write the joint posterior probability for the observed position of a star, satellite, and parent potential given the data as:
\begin{align}
	p(\Phi, \W, \W_s, \tau \given \D, \D_s) &= \frac{1}{\mathcal{Z}} p(\D \given \W) p(\D_s \given \W_s) 
												       p(\W \given \W_s, \tau, \Phi) 
												       p(\Phi) p(\tau)
\end{align}
where the factor $\mathcal{Z}$ only depends on the properties of the data (the \emph{evidence}), and is thus constant when varying the model parameters. We assume that the observational errors are Gaussian in heliocentric coordinates:
\begin{align}
	p(\D \given \W) &= \mathcal{N}(\W \given \D, \bSigma)\\
	p(\D_s \given \W_s) &= \mathcal{N}(\W_s \given \D_s, \bSigma_s)
\end{align}
where the covariance matrices $\bSigma^{(i)}$ and $\bSigma_s$ represent the observational variances on the observed 6D position of star and the satellite, respectively. 

The disruption of a satellite galaxy due to tidal shocking [...] Thus, we model the present-day position and velocity of the disrupted star as the result of sampling from isotropic, log-normal distributions centered on the instantaneous tidal radius, $R_{\rm tide}$, and velocity dispersion, $\sigma_v$, of the satellite at some time $\tau$ in the past, evolved to present day in the parent potential, $\Phi$. In this model, we neglect the effect of self-interaction with the halo of the progenitor [TODO: maybe we need to justify this, add APW's figure from paper 1 ref. response? refer to another section]. The likelihood of a star's position today is computed by first transforming the positions to Galactocentric, cartesian coordinates and integrating the orbits of the satellite and star backwards for several Gigayears in the parent potential, $\Phi$, treating both as test particles. The distance, $R$, and velocity, $V$, of the star relative to the satellite are computed at each time-step, and the likelihood is computed as the product of an isotropic, log-normal centered on the tidal radius with an isotropic, log-normal centered on the velocity dispersion of the satellite:
\begin{align}
	p(\W \given \W_s, \tau, \Phi) &= p(\X \given \X_s, \tau, \Phi) \jac{(l,b,D,\mu_l,\mu_b,v_r)}\\
	p(\X \given \X_s, \tau, \Phi) &= p(\ln R, \ln V \given \tau, \Phi) \jac{(x,y,z,v_x,v_y,v_z)}\label{eq:int} \\
	p(\ln R, \ln V \given \tau, \Phi) &= \mathcal{N}(\ln R(\tau) | \ln \rtide(\tau))\mathcal{N}(\ln V(\tau) | \ln \sigma_v(\tau))
\end{align}
where in eq.~\ref{eq:int} we have implicitly integrated over the two angle coordinates associated with $\ln R$ and the two angles associated with $\ln V$.

% , but the vectors $\D$ and $\W$ [exist] in different coordinate systems --- heliocentric spherical and Galactocentric cartesian, respectively --- so we must also track the Jacobian that [describes] this transformation:

In general, $p(\tau)$ is likely conditional on the observed orbit of the progenitor but this is neglected and assumed to be uniform over the entire interaction history between the satellite and parent.

% \jac{(x,y,z)}




%, and the tildes denote a conversion from spherical, heliocentric to cartesian, Galactocentric coordinates. 
%Thus, 
%\begin{align}
%	p(\Phi, \W, \W_s, \bSigma_s, \tau \given \D, \D_s) &= \prod_i p(\Phi, \W^{(i)}, \W_s, \bSigma_s, \tau^{(i)} \given \D^{(i)}, \D_s)\\
%	p &= \prod_i p^{(i)}\\
%	\ln p &= \sum_i \ln p^{(i)}\\
%	\ln p = \sum_i [\ln\mathcal{N}(\W^{(i)} \given \D^{(i)}, \bsigma^{(i)}) &+ \ln\mathcal{N}(\W_s \given \D_s, \bsigma_s) + \ln\mathcal{N}(\W^{(i)}(\tau^{(i)}) \given \W_s(\tau^{(i)}), \bSigma_s(\tau^{(i)}))]
%\end{align}

We first consider the case of a spherical Gaussian progenitor described by a single (time-dependent) spatial scale -- the tidal radius, $r_{\mathrm{tide}}(t)$ -- and a single velocity scale -- the velocity dispersion, $\sigma_\mathrm{v}$, with a \emph{perfectly} observed position. Under these assumptions, the posterior can be written
\begin{align}
	p(\Phi, \W^{(i)}, \tau^{(i)} \given \D^{(i)}) &\propto 
		p(\D^{(i)} \given \W^{(i)}) p(\W^{(i)} \given \W_s, \bSigma_s, \tau^{(i)})\\
\end{align}
and thus the $\ln$-posterior is
\begin{align}
	\ln p(\Phi, \W^{(i)}, \tau^{(i)} \given \D^{(i)}) &= \ln p(\D^{(i)} \given \W^{(i)}) + \ln p(\W^{(i)} \given \W_s, \bSigma_s, \tau^{(i)}).
\end{align}
The first term on the RHS -- from the observed positions of the stars	-- can be expanded to read
\begin{align}
	\ln p(\D^{(i)} \given \W^{(i)}) &= K - \frac{1}{2}\sum_j \left[ 2\ln\sigma_j + \left(\frac{X^{(i)}_j-D^{(i)}_j}{\sigma^{(i)}_j}\right)^2\right].
\end{align}
The second term -- quantifying how well the potential ``returns'' the stars to the progenitor -- reads as follows for the case of a spherical progenitor:
\begin{align}
	\ln p(\W^{(i)} \given \W_s, \bSigma_s, \tau^{(i)}) &= M - \frac{1}{2}\ln(\left\vert\bSigma_s(\tau^{(i)})\right\vert) - \frac{1}{2}\langle \widetilde{\W}-\widetilde{\W}_s, \bSigma_s^{-1}(\widetilde{\W}-\widetilde{\W}_s) \rangle\\
	&= M - \left[3(\ln r_{\rm tide} + \ln \sigma_v) - \frac{1}{2}\left( \frac{({\boldsymbol r}^{(i)}-{\boldsymbol r}_s)^2}{r_{\rm tide}^2} + \frac{ ({\boldsymbol v}^{(i)}-{\boldsymbol v}_s)^2}{\sigma_v^2} \right)\right]_{t=\tau^{(i)}}\label{eq:ln_like}
\end{align}
where $\boldsymbol r$,$\boldsymbol v$ are the (cartesian) positions and velocities integrated \emph{backwards} in potential $\Phi$ from their present-day positions $\widetilde{\W}$. The whole expression Eq.~\ref{eq:ln_like} is evaluated at the time unbound, $\tau^{(i)}$.

%\begin{align}
%	= K - \frac{1}{2}\sum_j \left[ 2\ln\sigma_j + \left(\frac{X^{(i)}_j-D^{(i)}_j}{\sigma^{(i)}_j}\right)^2\right]&\notag \\
%	- \frac{1}{2}\ln(\left\vert\bSigma_s(\tau^{(i)})\right\vert) - \frac{1}{2}&\langle \W-\W_s, \bSigma_s^{-1}(\W-\W_s) \rangle\\
%	= K - \frac{1}{2}\sum_j \left[ 2\ln\sigma_j + \left(\frac{X^{(i)}_j-D^{(i)}_j}{\sigma^{(i)}_j}\right)^2\right]&\notag \\
%	- \frac{1}{2}\ln(\left\vert\bSigma_s(\tau^{(i)})\right\vert) - \frac{1}{2}&\langle \W-\W_s, \bSigma_s^{-1}(\W-\W_s) \rangle
%\end{align}

\section{Marginalize over $\tau$}
\begin{align}
	p(\Phi, \W^{(i)}, \tau^{(i)} \given \D^{(i)}) &\propto 
		p(\D^{(i)} \given \W^{(i)}) p(\W^{(i)} \given \W_s, \bSigma_s, \tau^{(i)})\\
	p(\Phi, \W^{(i)} \given \D^{(i)}) &= \int p(\Phi, \W^{(i)}, \tau^{(i)} \given \D^{(i)}) p(\tau^{(i)}) d\tau^{(i)}
\end{align}	
\begin{align}
	p(\Phi, \W^{(i)} \given \D^{(i)}) &\propto p(\D^{(i)} \given \W^{(i)}) \int p(\W^{(i)} \given \W_s, \bSigma_s, \tau^{(i)}) p(\tau^{(i)}) d\tau^{(i)}\\
	&\propto p(\D^{(i)} \given \W^{(i)}) \sum_j p(\W^{(i)} \given \W_s, \bSigma_s)|_{\tau^{(i)}_j} \Delta t
\end{align}
\begin{align}
	\ln p^{(i)} &\propto \ln p(\D^{(i)} \given \W^{(i)}) + \ln \sum_j p(\W^{(i)} \given \W_s, \bSigma_s, \tau^{(i)}_j) \Delta t\\
	&\propto \ln p(\D^{(i)} \given \W^{(i)}) + \ln \sum_j \mathcal{N}(\W^{(i)}( \tau^{(i)}_j) \given \W_s( \tau^{(i)}_j), \bSigma_s( \tau^{(i)}_j)) \Delta t\\
\end{align}


\end{document}
