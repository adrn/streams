\documentclass[letterpaper,12pt,preprint]{aastex}

% packages
\usepackage{amssymb,amsmath, amsbsy}

% commands
\newcommand{\given}{\,|\,}
\newcommand{\dd}{\mathrm{d}}
\newcommand{\transpose}[1]{{#1}^{\mathsf{T}}}
\newcommand{\inverse}[1]{{#1}^{-1}}

% bold X and D
\newcommand{\D}{{\bf D}}
\newcommand{\X}{{\bf X}}
\newcommand{\bSigma}{{\bf \Sigma}}
\newcommand{\bsigma}{\boldsymbol\sigma}

\begin{document}

\title{Back-integration likelihood}
\author{Adrian M. Price-Whelan}

Starting with notation, take ${\bf D}^{(i)}$ to be the observed heliocentric 6D position of the $i$th star -- e.g., the measured $l$, $b$, $D$, $\mu_l$, $\mu_b$, $v_r$ -- and ${\bf X}^{(i)}$ to be the corresponding \emph{true} 6D position of the same star. We assume that this star was once part of a progenitor system -- e.g., satellite galaxy -- presently at observed position ${\bf D}_s$ (with true position, ${\bf X}_s$ and shape described by a covariance matrix, $\bSigma_s$) that is tidally disrupting and forming a cold debris structure in the potential, $\Phi$, of some parent galaxy. We then model the measured position of a star today as the result of sampling from a time-dependent 6D distribution -- the progenitor system -- at some time $\tau^{(i)}$ in the past, evolved to present day in the parent potential, neglecting the effect of self-interaction with the halo of the progenitor. We model the progenitor as a 6D Gaussian with mean vector given by the position of the satellite at any given time, ${\bf X}_s(t)$, and with shape defined by a time-dependent covariance matrix $\bSigma_s(t)$ (the subscript $s$ refers to satellite or progenitor). With these assumptions we can write the posterior for a single star as:
\begin{align}
	p(\Phi, \X^{(i)}, \X_s, \bSigma_s, \tau^{(i)} \given \D^{(i)}, \D_s) &= 
		\frac{1}{Z} p(\D^{(i)} \given \X^{(i)}) p(\D_s \given \X_s) p(\X^{(i)} \given \X_s, \bSigma_s, \tau^{(i)}, \Phi)\\
	p(\D^{(i)} \given \X^{(i)}) &= \mathcal{N}(\X^{(i)} \given \D^{(i)}, \bsigma^{(i)})\\
	p(\D_s \given \X_s) &= \mathcal{N}(\X_s \given \D_s, \bsigma_s)\\
	p(\X^{(i)} \given \X_s, \bSigma_s, \tau^{(i)}, \Phi) &= \mathcal{N}(\widetilde{\X}^{(i)}(\tau^{(i)}) \given \widetilde{\X}_s(\tau^{(i)}), \bSigma_s(\tau^{(i)}))
\end{align}
where the vectors $\bsigma^{2}$ and $\bsigma_s^2$ represent the (assumed) Gaussian, uncorrelated observational variances on the observed 6D position of star and the satellite, respectively.
%, and the tildes denote a conversion from spherical, heliocentric to cartesian, Galactocentric coordinates. 
%Thus, 
%\begin{align}
%	p(\Phi, \X, \X_s, \bSigma_s, \tau \given \D, \D_s) &= \prod_i p(\Phi, \X^{(i)}, \X_s, \bSigma_s, \tau^{(i)} \given \D^{(i)}, \D_s)\\
%	p &= \prod_i p^{(i)}\\
%	\ln p &= \sum_i \ln p^{(i)}\\
%	\ln p = \sum_i [\ln\mathcal{N}(\X^{(i)} \given \D^{(i)}, \bsigma^{(i)}) &+ \ln\mathcal{N}(\X_s \given \D_s, \bsigma_s) + \ln\mathcal{N}(\X^{(i)}(\tau^{(i)}) \given \X_s(\tau^{(i)}), \bSigma_s(\tau^{(i)}))]
%\end{align}

We first consider the case of a spherical Gaussian progenitor described by a single (time-dependent) spatial scale -- the tidal radius, $r_{\mathrm{tide}}(t)$ -- and a single velocity scale -- the velocity dispersion, $\sigma_\mathrm{v}$, with a \emph{perfectly} observed position. Under these assumptions, the posterior can be written
\begin{align}
	p(\Phi, \X^{(i)}, \tau^{(i)} \given \D^{(i)}) &\propto 
		p(\D^{(i)} \given \X^{(i)}) p(\X^{(i)} \given \X_s, \bSigma_s, \tau^{(i)})\\
\end{align}
and thus the $\ln$-posterior is
\begin{align}
	\ln p(\Phi, \X^{(i)}, \tau^{(i)} \given \D^{(i)}) &= \ln p(\D^{(i)} \given \X^{(i)}) + \ln p(\X^{(i)} \given \X_s, \bSigma_s, \tau^{(i)}).
\end{align}
The first term on the RHS -- from the observed positions of the stars	-- can be expanded to read
\begin{align}
	\ln p(\D^{(i)} \given \X^{(i)}) &= K - \frac{1}{2}\sum_j \left[ 2\ln\sigma_j + \left(\frac{X^{(i)}_j-D^{(i)}_j}{\sigma^{(i)}_j}\right)^2\right].
\end{align}
The second term -- quantifying how well the potential ``returns'' the stars to the progenitor -- reads as follows for the case of a spherical progenitor:
\begin{align}
	\ln p(\X^{(i)} \given \X_s, \bSigma_s, \tau^{(i)}) &= M - \frac{1}{2}\ln(\left\vert\bSigma_s(\tau^{(i)})\right\vert) - \frac{1}{2}\langle \widetilde{\X}-\widetilde{\X}_s, \bSigma_s^{-1}(\widetilde{\X}-\widetilde{\X}_s) \rangle\\
	&= M - \left[3(\ln r_{\rm tide} + \ln \sigma_v) - \frac{1}{2}\left( \frac{({\boldsymbol r}^{(i)}-{\boldsymbol r}_s)^2}{r_{\rm tide}^2} + \frac{ ({\boldsymbol v}^{(i)}-{\boldsymbol v}_s)^2}{\sigma_v^2} \right)\right]_{t=\tau^{(i)}}\label{eq:ln_like}
\end{align}
where $\boldsymbol r$,$\boldsymbol v$ are the (cartesian) positions and velocities integrated \emph{backwards} in potential $\Phi$ from their present-day positions $\widetilde{\X}$. The whole expression Eq.~\ref{eq:ln_like} is evaluated at the time unbound, $\tau^{(i)}$.

%\begin{align}
%	= K - \frac{1}{2}\sum_j \left[ 2\ln\sigma_j + \left(\frac{X^{(i)}_j-D^{(i)}_j}{\sigma^{(i)}_j}\right)^2\right]&\notag \\
%	- \frac{1}{2}\ln(\left\vert\bSigma_s(\tau^{(i)})\right\vert) - \frac{1}{2}&\langle \X-\X_s, \bSigma_s^{-1}(\X-\X_s) \rangle\\
%	= K - \frac{1}{2}\sum_j \left[ 2\ln\sigma_j + \left(\frac{X^{(i)}_j-D^{(i)}_j}{\sigma^{(i)}_j}\right)^2\right]&\notag \\
%	- \frac{1}{2}\ln(\left\vert\bSigma_s(\tau^{(i)})\right\vert) - \frac{1}{2}&\langle \X-\X_s, \bSigma_s^{-1}(\X-\X_s) \rangle
%\end{align}

\section{Marginalize over $\tau$}
\begin{align}
	p(\Phi, \X^{(i)}, \tau^{(i)} \given \D^{(i)}) &\propto 
		p(\D^{(i)} \given \X^{(i)}) p(\X^{(i)} \given \X_s, \bSigma_s, \tau^{(i)})\\
	p(\Phi, \X^{(i)} \given \D^{(i)}) &= \int p(\Phi, \X^{(i)}, \tau^{(i)} \given \D^{(i)}) p(\tau^{(i)}) d\tau^{(i)}
\end{align}	
\begin{align}
	p(\Phi, \X^{(i)} \given \D^{(i)}) &\propto p(\D^{(i)} \given \X^{(i)}) \int p(\X^{(i)} \given \X_s, \bSigma_s, \tau^{(i)}) p(\tau^{(i)}) d\tau^{(i)}\\
	&\propto p(\D^{(i)} \given \X^{(i)}) \sum_j p(\X^{(i)} \given \X_s, \bSigma_s)|_{\tau^{(i)}_j} \Delta t
\end{align}
\begin{align}
	\ln p^{(i)} &\propto \ln p(\D^{(i)} \given \X^{(i)}) + \ln \sum_j p(\X^{(i)} \given \X_s, \bSigma_s, \tau^{(i)}_j) \Delta t\\
	&\propto \ln p(\D^{(i)} \given \X^{(i)}) + \ln \sum_j \mathcal{N}(\X^{(i)}( \tau^{(i)}_j) \given \X_s( \tau^{(i)}_j), \bSigma_s( \tau^{(i)}_j)) \Delta t\\
\end{align}


\end{document}
