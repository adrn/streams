%\documentclass{emulateapj}
\documentclass[letterpaper,12pt,preprint]{aastex}

% packages
\usepackage{amssymb,amsmath, amsbsy}
\usepackage{booktabs}

% commands
\newcommand{\project}[1]{\textsl{#1}}
\newcommand{\gaia}{\project{Gaia}~}
\newcommand{\spitzer}{\project{Spitzer}~}
\newcommand{\given}{\,|\,}
\newcommand{\dd}{\mathrm{d}}
\newcommand{\transpose}[1]{{#1}^{\mathsf{T}}}
\newcommand{\inverse}[1]{{#1}^{-1}}
\newcommand{\jac}[1]{\left\vert \J{#1} \right\vert}

\newcommand{\D}{{\bf D}}
\newcommand{\W}{{\bf W}}
\newcommand{\X}{{\bf X}}
\newcommand{\J}{{\boldsymbol J}}
\newcommand{\bSigma}{{\bf \Sigma}}
\newcommand{\bsigma}{\boldsymbol\sigma}
\newcommand{\rtide}{R_{\rm tide}}


\begin{document}

\title{Inferring the gravitational potential of the Milky Way with few precisely measured stars}
\author{Adrian M. Price-Whelan, DWH, KVJ, others?}

\begin{abstract}
%Informative phase-space structures, and in particular those created by
%disrupting stellar systems or revealed by chemical tags, may deliver
%very precise measures of the gravitational potential in the Milky Way
%Halo.  Here we show that even a single pair of stars---two stars that
%are known (for some non-kinematic reason) to be likely to be
%associated with one another at birth---could provide a significant
%constraint on the potential.  The inference is based on a
%probabilistic generative model---a likelihood function and priors over
%nuisance parameters---that evolves a past putative origin disruption
%event forward in time to the present day.  The time, six-volume, and
%phase-space location of the origin event are nuisance parameters in
%the model and marginalized away.  The method makes no assumption of
%integrability, works with finite or even large observational
%uncertainties, does not require all dimensions of phase space to be
%observed, handles non-zero probability that the two stars are not in
%fact associated, and generalizes naturally to larger numbers of stars
%and multiple independent structures.  Applications to the GD-1 cold
%stellar stream and current surveys for RR Lyrae stars in the Halo are
%discussed.
TODO...
\end{abstract}

\keywords{
  Galaxy: structure
  ---
  Galaxy: halo
  ---
  cosmology: dark matter
}

\section{Introduction}

[Cold streams contain tons of information.]

[GD-1 was fit as if it highlighted an orbit.  That is known to be
  wrong.]

[In the halo, any phase-space structure might be long-lived.  Chemical
  tagging could in principle illuminate it.]

[Chemical tagging will always only give probabilistic information.
  But what if it delivers somewhat confident information about small
  stellar families in the halo?]

[Actions, angles, integrable, not, etc.  Can integrate in any
  potential, even a time-varying one. We use this to
  ``run the clock back''.  Why is an objective based on past collisions not the
  same as assuming that the stars are on the same orbit?]

[Sometimes stellar samples might get extremely small.  For example,
  \gaia\ might find some very low-mass cold stellar streams.  For
  another, there might only be a coupld RRL stars in a particular
  stream, but those RRL stars might be individually extremely well
  measured in six-space.]

[Obviously a small stellar sample can only tell you a small number of
  things about the Milky Way potential.  The idea is---in the long
  run---to construct likelihood functions for many small stellar
  samples and in the end do inference by multiplying them together.
  Duh!]

[Densities are hard to match -- need a background model, way to go from luminosity to density, needs assumptions about the mass function and structure of the satellite\

[Cold vs. Hot streams in the context of observational errors]

[We properly handle observational errors]

\section{The model}

Take $\D$ to be the observed, heliocentric 6D position of a star --- e.g., the measured position on the sky, ($l$, $b$); distance, $D$; proper motions, ($\mu_l$, $\mu_b$); and line-of-sight velocity, $v_r$ --- and take $\W$ to be the corresponding \emph{true}, 6D position of the same star in the same coordinate and reference frame. Assume that we have determined that this star was once part of a progenitor system (e.g., satellite galaxy) tidally disrupting and forming a cold debris structure in the potential, $\Phi$, of some parent galaxy. The position of the satellite is observed to be at heliocentric 6D position $\D_s$, with true position $\W_s$, where the subscript $s$ refers to the satellite. The true, 6D spherical, heliocentric positions of the star and satellite -- $\W$ and $\W_s$ -- have corresponding 6D cartesian, Galactocentric coordinate vectors $\X$ and $\X_s$. Initially, we assume the present-day mass of the satellite, $M_s$, and velocity dispersion, $\sigma_v$, have been measured perfectly (but we will return to this later). With this notation, we write the joint posterior probability for the observed position of a star, satellite, and parameters for the parent potential as:
\begin{align}
	p(\Phi, \W, \W_s, \tau \given \D, \D_s) &= \frac{1}{\mathcal{Z}} p(\D \given \W) p(\D_s \given \W_s) 
												       p(\W \given \W_s, \tau, \Phi) 
												       p(\Phi) p(\tau)
\end{align}
where the factor $\mathcal{Z}$ only depends on the properties of the data (the \emph{evidence}), and is thus constant when varying the model parameters. We assume that the observational errors are Gaussian in heliocentric coordinates:
\begin{align}
	p(\D \given \W) &= \mathcal{N}(\W \given \D, \bSigma)\\
	p(\D_s \given \W_s) &= \mathcal{N}(\W_s \given \D_s, \bSigma_s)
\end{align}
where the covariance matrices $\bSigma$ and $\bSigma_s$ specify the observational variances on the observed 6D position of star and the satellite, respectively. 

The disruption of a satellite galaxy due to tidal shocking [...] Thus, we model the present-day position and velocity of the disrupted star as the result of sampling from isotropic, log-normal distributions centered on the instantaneous tidal radius, $R_{\rm tide}$, and velocity dispersion, $\sigma_v$, of the satellite at some time $\tau$ in the past, evolved to present day in the parent potential, $\Phi$. In this model, we neglect the effect of self-interaction with the halo of the progenitor [TODO: maybe we need to justify this, add APW's figure from paper 1 ref. response? refer to another section]. The likelihood of a star's position today is computed by first transforming the positions to Galactocentric, cartesian coordinates and integrating the orbits of the satellite and star backwards for several Gigayears in the parent potential, $\Phi$, treating both as test particles. The distance, $R$, and velocity, $V$, of the star relative to the satellite are computed at each time-step, and the likelihood is computed as the product of an isotropic, log-normal centered on the tidal radius with an isotropic, log-normal centered on the velocity dispersion of the satellite:
\begin{align}
	p(\W \given \W_s, \tau, \Phi) &= p(\X \given \X_s, \tau, \Phi) \jac{(l,b,D,\mu_l,\mu_b,v_r)}\\
	p(\X \given \X_s, \tau, \Phi) &= p(\ln R, \ln V \given \tau, \Phi) \jac{(x,y,z,v_x,v_y,v_z)}\label{eq:int} \\
	p(\ln R, \ln V \given \tau, \Phi) &= \mathcal{N}(\ln R(\tau) | \ln \rtide(\tau))\mathcal{N}(\ln V(\tau) | \ln \sigma_v(\tau))
\end{align}
where in eq.~\ref{eq:int} we have implicitly integrated over and neglected the two angle coordinates associated with $\ln R$ and the two angles associated with $\ln V$.

% , but the vectors $\D$ and $\W$ [exist] in different coordinate systems --- heliocentric spherical and Galactocentric cartesian, respectively --- so we must also track the Jacobian that [describes] this transformation:

In general, $p(\tau)$ is likely conditional on the observed orbit of the progenitor---stars are preferentially stripped at pericentric passages \citep{??}---but this is neglected and assumed to be uniform over the entire interaction history between the satellite and parent. Similarly, the priors on potential parameters, $p(\Phi)$, are taken to be uniform.

\section{Experiments}
We conduct several tests of this method [...]. For all tests in this article, we model the potential of the parent galaxy as a three-component sum of a Miyamoto-Nagai disk \citep{}, Hernquist bulge \citep[spheroid][]{}, and a triaxial, logarithmic halo \citep[e.g.,][]{law10}:
\begin{align}
	&\Phi_{\rm disk}(R,z) = -\frac{GM_{\rm disk}}{\sqrt{R^2 + (a + \sqrt{z^2 + b^2})^2}}\\
	&\Phi_{\rm spher}(r) = -\frac{GM_{\rm spher}}{r + c}\\
	&\Phi_{halo}(x,y,z) = v_{h}^2 \ln(C_1 x^2 + C_2 y^2 + C_3 xy + (z/q_z)^2 + r_h^2)
\end{align}
where $C_1$, $C_2$, and $C_3$ are combinations of the $x$ and $y$ axis
ratios ($q_1$, $q_2$) and orientation of the halo with respect to the
baryonic disk ($\phi$):
\begin{align}
  C_1 &= \frac{\cos^2\phi}{q_1^2} + \frac{\sin^2\phi}{q_2^2}\\
  C_2 &= \frac{\sin^2\phi}{q_1^2} + \frac{\cos^2\phi}{q_2^2}\\
  C_3 &= 2\sin\phi\cos\phi \left(q_1^{-2} - q_2^{-2}\right).
\end{align}

[TODO: explain simulated data we use - KVJ are you ok releasing data from your 2.5e6-2.5e9 Sgr sims? Explain that we use the LM10 potential in Table~\ref{tbl:potential} for the sims]. 

When recovering the potential, we hold fixed the disk and spheroid parameters (see Table~\ref{tbl:potential}), along with two halo parameters: the scale radius, $r_h$, and one of the flattening components in the x-y plane, $q_2$. These halo parameters are degenerate with combinations of the other parameters --- e.g., setting $q_1=2.0$, $q_2=1.0$, $\phi=0^\circ$ is equivalent to setting $q_1=1.0$, $q_2=2.0$, $\phi=90^\circ$. The priors on the remaining halo parameters are taken to be uniform over a conservative domain of realistic values: for $v_h$, 100-200~kms$^{-1}$ corresponds to a range in solar circular velocities from $\sim$210-250 kms$^{-1}$ (holding other parameters fixed); the axis ratios [...]; and $\phi$ is restricted to $\pm45^\circ$ around the value measured by LM10 ($\phi = 97^\circ$).

\begin{table}[h]
\begin{center}
	\begin{tabular}{l c c c} \toprule
		{\bf Component} & {\bf Parameter} & {\bf Value} & {\bf Prior} \\\toprule
		disk & $M_{\rm disk}$ & $1.0\times10^{11}M_\odot$ & -- \\ 
		& $a$ & 6.5 kpc & --\\
		& $b$ & 0.26 kpc & --\\
		\midrule
		spheroid & $M_{\rm spher}$ & $3.4\times10^{10}M_\odot$ & --\\ 
		& $c$ & 0.7 kpc & --\\
		\midrule
		halo & $v_h$ & -- & $\mathcal{U}(100,200)$ km $\mathrm{s}^{-1}$ \\
		& $q_1$ & -- & $\mathcal{U}(1,2)$\\
		& $q_2$ & 1.0 & --\\
		& $q_z$ & -- & $\mathcal{U}(1,2)$\\
		& $\phi$ & -- & $\mathcal{U}(52,142)$ deg\\
		& $r_h$ & 12 kpc & --\\
		\bottomrule
		\hline
	\end{tabular}
	\caption{Parameter values used ...\label{tbl:potential}}
\end{center}
\end{table}

\subsection{\gaia + \spitzer}
64 stars in Sgr-like stream

\subsection{\spitzer + current data}
64 stars in Sgr-like stream

\subsection{\spitzer + current data, no progenitor}
32 stars in Orphan-like stream, XX stars in GD-1 -like stream

\section{Discussion}

[Cold vs. Hot streams]

[bow-tie structure, Kuepper work, and so on]

[What's unrealistic about what we have done?  The kinematic model of
  the progenitor is exceedingly simplistic; why are we okay with it
  nonetheless but what could we do to improve it?  Related to this,
  the progenitor doesn't shrink with time (as it should).  What's up
  with that?]

[We did not assume the potential is static in time.  At least not in
  general; nothing in the method requires or suggests this.]

[Issues about computation and scaling.]

[Nothing special about streams; will work on any structures, including
  shells and the like.]

[Don't eschew low-dimensional constraints on the potential!]

\bibliographystyle{apj}
\bibliography{refs}

\acknowledgements
Binney, Rix, Sanders

\end{document}
