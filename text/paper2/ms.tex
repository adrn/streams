%\documentclass{emulateapj}
\documentclass[letterpaper,12pt,preprint]{aastex}

% packages
\usepackage{amssymb,amsmath, amsbsy}
\usepackage{booktabs}

% commands
\newcommand{\project}[1]{\textsl{#1}}
\newcommand{\gaia}{\project{Gaia}~}
\newcommand{\spitzer}{\project{Spitzer}~}
\newcommand{\given}{\,|\,}
\newcommand{\dd}{\mathrm{d}}
\newcommand{\transpose}[1]{{#1}^{\mathsf{T}}}
\newcommand{\inverse}[1]{{#1}^{-1}}
\newcommand{\jac}[1]{\left\vert \J{#1} \right\vert}

\newcommand{\D}{{\bf D}}
\newcommand{\W}{{\bf W}}
\newcommand{\X}{{\bf X}}
\newcommand{\J}{{\boldsymbol J}}
\newcommand{\bSigma}{{\bf \Sigma}}
\newcommand{\bsigma}{\boldsymbol\sigma}
\newcommand{\rtide}{r_{\rm tide}}
\newcommand{\bs}{\boldsymbol}

\newcommand{\paperone}{Paper 1}

\begin{document}

\title{Inferring the gravitational potential of the Milky Way with a few precisely measured stars}
\author{Adrian M. Price-Whelan\altaffilmark{\colum,\adrn}, 
	    David W. Hogg\altaffilmark{\nyu,\mpia}, 
	    Kathryn V. Johnston\altaffilmark{\colum}, 
	    David Hendel\altaffilmark{\colum}}

% Affiliations
\newcommand{\colum}{1}
\newcommand{\adrn}{2}
\newcommand{\nyu}{3}
\newcommand{\mpia}{4}
\altaffiltext{\colum}{Department of Astronomy, 
		              Columbia University, 
		              550 W 120th St., 
		              New York, NY 10027, USA}
\altaffiltext{\adrn}{To whom correspondence should be addressed: adrn@astro.columbia.edu}
\altaffiltext{\nyu}{Center for Cosmology and Particle Physics,
                      Department of Physics, New York University,
                      4 Washington Place, New York, NY, 10003, USA}
\altaffiltext{\mpia}{Max-Planck-Institut f\"ur Astronomie,
                     K\"onigstuhl 17, D-69117 Heidelberg, Germany}

\begin{abstract}
% Context
 [this needs to get shorter, but]
The dark-matter halo of the Milky Way is expected to be triaxial on large scales
  and filled with substructure on all smaller scales.
It is hoped that streams or shells of stars produced by tidal disruption
  will produce very precise measures of the gravitational potential,
  to test these predictions.
% Aims
We develop a justified method for inferring the gravitational potential
  using precise measurements of stars in tidal streams
  based on an approximate generative model.
Our method does not rely on integrability of the potential;
  we don't expect there to be any good approximation to the Milky Way potential that is integrable.
% Methods: 
Our model is probabilistic, with a likelihood function and priors on the parameters.
It is based on the idea that the stars in a tidal structure
  were close in phase space at some point in the past;
  model parameters include a ``disruption time'' [APW adjust language] for each star,
  and phase-space coordinates for the progenitor.
It accounts properly for finite observational uncertainties and missing data.
We test our method with simulated data in a non-integrable triaxial potential,
  with realistic observational uncertainties for the next generation of surveys of RR Lyrae stars and the like.
Importantly, the fake data were generated with an n-body simulation,
  not with the approximate model underlying our likelihood function.
% Results:
We find that even with a small number (4 to 16) of well-measured stars,
  we can precisely infer properties of the triaxial potential,
  missing phase-space coordinates of the stars,
  and the location of the progenitor object.
The method generalizes to to encompass mixtures of independent debris structures,
  or situations in which star membership in structures is unknown.
[Get more quantitative here?]
% Conclusions: 
\end{abstract}

\keywords{
  cosmology: dark matter
  ---
  Galaxy: structure
  ---
  Galaxy: halo
}

\section{Introduction}

Cosmological simulations of galaxy formation in the $\Lambda$CDM paradigm predict dark matter halos that (1) are permeated with substructure on many scales, (2) are triaxial in shape, and (3) have density profiles and shapes that vary with radius \citep{dubinski91, jing02, kuhlen07, veraciro11}. Dark-matter-only simulations produce triaxial halos \citep{jing02} with large density fluctuations \citep{zemp09}. Including baryons tends to soften the triaxiality and graininess in the inner galaxy through a combination of dissipative infall \citep{dubinski94} or cooling \citep{bryan13}. These processes combined with the gravity from a baryonic disk (ellipsoid) can act to make the inner halo more oblate (spherical), however these processes do not erase the clumpy, triaxial nature of the outer halo \citep[e.g.,][]{pontzen12}. This leads to radially-dependent axis ratios and orientation, density profile, and smoothness, thus suggesting complicated, non-intregrable potential forms for Milky Way-like galaxies.

The bulk of the baryonic matter in galaxies spans roughly 5-10\% of the extent of the host dark matter halo. Hence, the major observable components of a galaxy are sensitive to the inner portion of its host halo's mass distribution. For example, the rotation curves of disk galaxies are excellent probes of the inner mass, especially since matter in disks can be assumed to move on near circular orbits. Conversely, measuring the dark matter distribution at large radii is complicated by the low density of visible tracers, observational difficulties, and [unclear what assumptions can be made about the integrability and smoothness of the mass distribution]. Around external galaxies, the extended mass distribution has been probed using a variety of approaches \citep[see][for a a complete and detailed review]{courteau13}. For example, the kinematics of tracer populations such as globular clusters or planetary nebulae can be used to derive mass estimates under the assumptions that these satellite systems are relaxed and well-mixed \citep[early investigations include][]{mendez01,cote03}. Simple, parameterized models of both the mass and orbit distribution have been simultaneously constrained using such data \citep[e.g.][]{napolitano11,deason12c}. 
%\citep{romanowsky03,douglas07,romanowsky09,napolitano11,lee11}
Alternatively, the statistical properties of gravitationally lensed background sources around a galaxy can be used to constrain the \emph{projected} shape, orientation, and radial profile of mass \citep[as done by the Lens Structure and Dynamics Survey described in][]{koopmans02}. Of course, lensing reconstructions can only be performed for galaxies which closely intersect our line of sight to background sources, but the advent of large photometric catalogues has allowed automatic searches for such chance alignments and significant increases in the number of objects studied in this way \citep[e.g. the Sloan Lens ACS Survey, see][]{bolton06}.

Within the Milky Way our unique vantage point allows us a three-dimensional view of the structure in own dark matter halo. Our proximity allows us to use individual stars as kinematic tracers and hence build much larger samples that probe deeper into the halo than the globular cluster and planetary nebula studies of external galaxies. For example, \cite{deason12a} used halo BHB stars selected from the Sloan Digital Sky Survey \cite[SDSS;][]{york00} as a random tracer population to measure the mass and slope of a simple, power-law fit to the Galactic potential. Such studies assume that the tracer orbits are randomly sampled from a smooth distribution function and phase mixed, however the existence of large-scale substructure in stellar streams and clouds is known to bias mass and velocity inferences that ignore this fact. Another approach is to exploit the non-random nature of the stellar populations in the halo. Tidal streams are dynamically cold systems --- debris typically have small distributions of energy and angular momentum --- and thus require orders of magnitude fewer tracers than a random sample to get constraints of comparable accuracy. For example, in the simplest case we might assume that debris stars are actually still on the same orbit as their progenitor system \citep[a \emph{wrong} assumption, see e.g.,][]{eyre2011}. Then we can imagine measuring the full-space velocities ${\bf v}$ at different points ${\bf x}$ in the structure (e.g., along a stream), which would give us a direct measure of differences in a potential, $\Phi$. 
%In practice, simple methods such as orbit fitting lead to systematic biases in inferred properties of the underlying potential because streams do not strictly delineate orbits [cite eyre binney work]. 

Since the initial discovery of tidal debris from Sgr, myriad other streams and debris systems have been identified in large photometric surveys, probing Galactocentric distances from $\sim$15-100 kpc. The known debris structures are $\gtrsim$10 kpc from the Sun, thus while overdensities in the halo are less contaminated by background, the constituent stars often have incomplete or poorly measured phase-space information. Perhaps the most rigorous method of modeling streams and the potential is to run full N-body simulations of satellite disruption and compare the density of observed stars to simulated stars in whatever coordinates are measured. A generative N-body model could naturally incorporate observational uncertainties, missing dimensions, and would, in principle, allow for studying any arbitrary time-dependent or non-integrable potential. However, such a model is presently computationally intractable. \cite{law10} demonstrated the power of this idea by running a grid of N-body models of the disruption of the Sagittarius (Sgr) dwarf spheroidal galaxy and matched their simulated streams to all Sgr data available at that time. By comparing the phase-space positions of their stream particles with the data, they were able to constrain the shape of the Milky Way potential out to $\sim$70 kpc and found that the best-fitting halo has a prolate, triaxial shape elongated in the Galactic Z direction, orthogonal to the plane of the disk. 
% (though, see ... for possible complications in the Sgr debris belokurov 2012?). 

The difficulty and computational costs associated with proper N-body modeling has motivated numerous efforts for developing approximate methods for constraining properties of the Galactic potential with cold debris structures. The simplest approximation is to fit a single orbit to observed debris \citep[e.g.,][]{koposov10, deg13}. This has the advantage of naturally incorporating observational errors, but is known to be invalid as debris does not exactly follow an orbit \citep[e.g.,][]{eyre11, sanders13a}. To account for this, methods have been proposed that model the debris as an orbit plus some dispersion or offset, either in phase space \citep[e.g.,][]{eyre09a, varghese11, kuepper12} or action-angle coordinates \citep{eyre11, sanders13b, bovy14, sanders14}. Other statistical methods have been proposed \citep[][]{penarrubia12, sanderson14} that may prove powerful when applied to, e.g., data from the \gaia mission, where full 6D coordinates will be known for large samples of stars in the halo --- and therefore many debris structures --- but stream membership is not known for all stars (however these statistical methods do not account for observational errors).

The potential of the inner galaxy is likely well-described by a (set of) nearly-integrable potentials, thus the dynamics of collisionless systems is plausibly captured by a transformation to some integral-of-motion-space, e.g., action-angle coordinates. However there is uncertainty in the exact shape, radial profile, and graininess of the outer halo. To test the extent to which the outer halo is well-fit by integrable potentials, we cannot use methods that \emph{assume} integrability. In this article, we present a fully probabilistic method for measuring properties of a potential that only relies on numerical orbit integration in ordinary phase-space. The model uses individual stars as constraints on the potential, and can therefore handle very small samples of stars. Generative models that instead rely on matching densities (in phase-space, action-space, or observed coordinates) must also include fine details about the progenitor (e.g., mass function or structure) and a model for the background of stars. Our method also naturally extends to using multiple debris structures \citep[see][who illustrate the power of using multiple streams to simultaneously constrain the potential]{deg14}.

Although the \gaia\ project will deliver hundreds of millions of stars useful for dynamical inference,
  it still might be that the most relevant stellar samples for inferring the Milky Way potential might be very small in the end.
One reason is that if distances turn out to be important, stars that produce good distance estimates might be much more
  valuable than typical stars in the sample.
At halo distances, \gaia's parallaxes are not expected to be better than the distance indicators produced by the best
  variable stars observed at the best wavelengths; for example, near-infrared observations of RR Lyrae stars
  can in principle deliver XX percent photometric distances at essentially all halo distances (CITATIONS).
These valuable stars are rare (and their abundances are age and metallicity dependent);
  there could be many cold structures in the Milky Way halo that are highly constraining
  on the potential in principle, but which contain only a few good distance-indicating members.
That said, it is not yet known what the trade-offs are between having many stars at low precision and a few at high precision,
  nor is it known how valuable distance information really is, when a structure contains many precisely observed members.

Another reason to think about small samples is more conceptual:
In dynamical inferences that go beyond simple virial or Jeans arguments,
  inference depends on the non-linear dynamical evolution of populations in phase space.
The nonlinearities, combined with heteroskedastic data and requirements to marginalize over phase-space distribution functions,
  lead naturally to situations in which some stars are \emph{far more informative} than others.
For a simple example, in a toy study of the Solar System,
  phase-space distribution marginalization was shown to make some planets are far more informative than others
  about the gravitational potential (CITE Bovy et al).
This is expected to be a generic feature in dynamical problems;
  it might be that a very small fraction of \gaia's catalog entries
  deliver a large fraction of the information about the Milky Way's potential and formation history.

In \citet[][hereafter \paperone]{apw13}, we introduced a simple method (\emph{Rewinder}) for using individual stars associated with tidal debris combined with knowledge about the mass and orbit of the progenitor to constrain properties of the host galaxy potential. The method exploits the relationship between the phase-space distribution of debris and (measurable) properties of the progenitor system (e.g., the tidal radius and escape velocity). In this paper, we investigate how these relationships scale with the mass and orbit of the progenitor and the quality of the data, then present a new probabilistic framework for simultaneously modeling stream debris and the parent potential based on the same arguments.

In Section~\ref{sec:sims} we describe a suite of N-body simulations that span a range of progenitor masses over two characteristic orbits. In Section~\ref{sec:} we 

\section{Simulations}\label{sec:sims}
[David H.]
We [...] N-body simulations run with an SCF code [...]. [describe simulations here] [Describe tub]

\section{Method}

\subsection{Notation}
Take $\D$ to be the observed, heliocentric 6D position of a star --- e.g., the measured position on the sky, ($l$, $b$); distance, $D$; proper motions, ($\mu_l$, $\mu_b$); and line-of-sight velocity, $v_r$ --- and take $\W$ to be the corresponding \emph{true}, 6D position of the star in the same coordinate and reference frame. Assume that we have determined that this star was once part of a progenitor system (e.g., satellite galaxy) with mass $M_p(t)$ and velocity dispersion $\sigma_p$ tidally disrupting and forming a cold debris structure in the potential, $\Phi$, of some parent galaxy. The position of the progenitor is observed to be at heliocentric 6D position $\D_p$, with true position $\W_p$ (where any subscript $p$ refers to the progenitor). The true 6D heliocentric spherical positions of the star and progenitor -- $\W$ and $\W_p$ -- have corresponding 6D Galactocentric cartesian coordinate vectors $\X$ and $\X_p$ comprised of 3D position and velocity vectors ($\bs{r}$,$\bs{v}$) and ($\bs{r}_p$,$\bs{v}_p$). 

\section{Scalings}

In \paperone, we define a set of coordinates relative to the instantaneous phase-space position of the progenitor, $(\bs{r}_{p},\bs{v}_{p})$, normalized by the instantaneous tidal radius and escape velocity. Here we (1) instead scale the velocity coordinates by the velocity dispersion of the satellite, (2) reduce the dimensionality by considering only the magnitude of these vectors, and (3) take the logarithm of the magnitude,
\begin{equation}
  R = \ln\left(\frac{\left\vert\bs{r} - \bs{r}_{p}\right\vert}{R_{\rm tide}}\right)\,\,\,\,,\,\,\,\,
  V = \ln\left(\frac{\left\vert\bs{v} - \bs{v}_{p}\right\vert}{\sigma_v}\right).
\end{equation}
Figure~\ref{fig:reldists} shows distributions of $R$ and $V$ at $t=t_{\rm ub}$ for all particles as a function of progenitor mass from the simulations described in Section~\ref{sec:sims}. In these scaled coordinates, the distributions 

a phase-space distance, $D_{ps}$, as the sum of the normalized, relative position, $\frac{\bs{r}-\bs{r}_p}{\rtide}$, and velocity, $\frac{\bs{v}-\bs{v}_p}{v_{\rm esc}}$, for a stream star. We minimized the generalized variance of the 6D distribution over each normalized coordinate evaluated at the time that each star has minimum $D_ps$. This simple idea proved quite powerful for constraining properties of the LM10 potential used simulated observations of their N-body simulation. However, this method does not properly incorporate observational uncertainties. 

Using the simulations described in Section~\ref{sec:sims}

\begin{figure}[h]
\begin{center}
% \includegraphics[width=\textwidth]{/Users/adrian/projects/streams/plots/paper2/rel_dists.pdf}
\caption{ TODO }\label{fig:reldist}
\end{center}
\end{figure}

%The core of this method provides a basis for the probabilistic model presented below. 

[Cold vs. Hot streams in the context of observational errors]

\subsection{Probabilistic model}
With this notation, we write the joint posterior probability for the observed position of a star, progenitor, and parent potential parameters as:
\begin{align}
	p(\Phi, \W, \W_p, \tau \given \D, \D_p) &= \frac{1}{\mathcal{Z}} p(\D \given \W) p(\D_p \given \W_p) 
												       p(\W \given \W_p, \tau, \Phi) 
												       p(\Phi) p(\tau)
\end{align}
where the factor $\mathcal{Z}$ only depends on the properties of the data (the \emph{evidence}), and is thus constant when varying the model parameters. We assume that the observational errors are Gaussian in heliocentric coordinates:
\begin{align}
	p(\D \given \W) &= \mathcal{N}(\W \given \D, \bSigma)\\
	p(\D_p \given \W_p) &= \mathcal{N}(\W_p \given \D_p, \bSigma_p)
\end{align}
where the covariance matrices $\bSigma$ and $\bSigma_p$ specify the observational uncertainties on the observed 6D position of star and the progenitor, respectively. 

The disruption of a satellite galaxy due to tidal shocking [...connect to Choi work some how?] We model the present-day position and velocity of a disrupted star as the result of sampling from isotropic, log-normal distributions in relative distance, $R=\left\vert \bs{r}-\bs{r}_p \right\vert$, and relative velocity, $V=\left\vert \bs{v}-\bs{v}_p \right\vert$, at the time the star comes unbound from the progenitor, evolved to present day in the parent potential, $\Phi$. The distributions over $\ln R$ and $\ln V$ are centered on the logarithm of the instantaneous tidal radius, $\rtide$, and velocity dispersion, $\sigma_v$, of the progenitor at time $\tau$: $\mathcal{N}(\ln R \given \ln\rtide)\vert_{t=\tau}$ and $\mathcal{N}(\ln V \given \ln \sigma_v)\vert_{t=\tau}$. In this model, we neglect the effect of self-interaction with the halo of the progenitor. 

The likelihood of a star's position today is computed by first transforming the observed position to Galactocentric cartesian coordinates and integrating the orbits of the progenitor and star backwards for several Gigayears in the parent potential, $\Phi$, treating both as test particles. The distance, $R$, and velocity, $V$, of the star relative to the progenitor are computed at each time-step, and the likelihood is computed as the product of the isotropic, log-normal centered on the logarithm of the instantaneous tidal radius with the isotropic, log-normal centered on the logarithm of the velocity dispersion of the progenitor:
\begin{align}
	p(\W \given \W_p, \tau, \Phi) &= p(\X \given \X_p, \tau, \Phi) \jac{(l,b,D,\mu_l,\mu_b,v_r)}_{t=\tau}\\
	p(\X \given \X_p, \tau, \Phi) &= p(\ln R, \ln V \given \tau, \Phi) \jac{(x,y,z,v_x,v_y,v_z)}_{t=\tau}\label{eq:int} \\
	p(\ln R, \ln V \given \tau, \Phi) &= \left[\mathcal{N}(\ln R | \ln \rtide, \sigma_R)\mathcal{N}(\ln V | \ln \sigma_v, \sigma_V)\right]_{t=\tau}
\end{align}
% , but the vectors $\D$ and $\W$ [exist] in different coordinate systems --- heliocentric spherical and Galactocentric cartesian, respectively --- so we must also track the Jacobian that [describes] this transformation:
where in eq.~\ref{eq:int} we have implicitly integrated over and neglected the two angle coordinates associated with $\ln R$ and the two angles associated with $\ln V$. 
\begin{align}
	\ln p(\ln R, \ln V \given \tau, \Phi) = -\ln(2\pi)-\ln(\sigma_R\sigma_V)-
		\ln\left[\left(\frac{R}{\rtide}\right)^{1/\sigma_R} \left(\frac{V}{\sigma_v}\right)^{1/\sigma_V}\right]
\end{align}
We then compute the marginal likelihood, $p(\W \given \W_p, \Phi)$, by integrating over all possible unbinding times,
\begin{align}
	p(\W \given \W_p, \Phi) = \int p(\W \given \W_p, \tau, \Phi)p(\tau) d\tau.
\end{align}
In general, $p(\tau)$ is likely conditional on the orbit of the progenitor --- stars are preferentially stripped at pericentric passages --- but this is neglected and is taken to be uniform over the entire interaction history between the progenitor and parent. 

\section{Experiments}
Initially we assume the present-day mass of the progenitor, $M_p$, and velocity dispersion, $\sigma_v$, have been measured perfectly and neglect mass-loss from the progenitor. 

For all tests in this article, we model the potential of the parent galaxy as a three-component sum of a Miyamoto-Nagai disk \citep{}, Hernquist bulge \citep[spheroid;][]{}, and a triaxial, logarithmic halo \citep[e.g.,][]{law10}:
\begin{align}
	&\Phi_{\rm disk}(R,z) = -\frac{GM_{\rm disk}}{\sqrt{R^2 + (a + \sqrt{z^2 + b^2})^2}}\\
	&\Phi_{\rm spher}(r) = -\frac{GM_{\rm spher}}{r + c}\\
	&\Phi_{halo}(x,y,z) = v_{h}^2 \ln(C_1 x^2 + C_2 y^2 + C_3 xy + (z/q_z)^2 + r_h^2)
\end{align}
where $C_1$, $C_2$, and $C_3$ are combinations of the $x$ and $y$ axis
ratios ($q_1$, $q_2$) and orientation of the halo with respect to the
baryonic disk ($\phi$):
\begin{align}
  C_1 &= \frac{\cos^2\phi}{q_1^2} + \frac{\sin^2\phi}{q_2^2}\\
  C_2 &= \frac{\sin^2\phi}{q_1^2} + \frac{\cos^2\phi}{q_2^2}\\
  C_3 &= 2\sin\phi\cos\phi \left(q_1^{-2} - q_2^{-2}\right).
\end{align}

\begin{figure}[h]
\begin{center}
% \includegraphics[width=\textwidth]{/Users/adrian/projects/streams/plots/paper2/potentials.pdf}
\caption{ TODO }\label{fig:potential}
\end{center}
\end{figure}
 
When recovering the potential, we hold fixed the disk and spheroid parameters (see Table~\ref{tbl:potential}), along with two halo parameters: the scale radius, $r_h$, and one of the flattening components in the x-y plane, $q_2$. These halo parameters are degenerate with combinations of the other parameters --- e.g., setting $q_1=2.0$, $q_2=1.0$, $\phi=0^\circ$ is equivalent to setting $q_1=1.0$, $q_2=2.0$, $\phi=90^\circ$. The priors on the remaining halo parameters are taken to be uniform over a conservative domain of realistic values: for $v_h$, 100-200~kms$^{-1}$ corresponds to a range in solar circular velocities from $\sim$210-250 kms$^{-1}$ (holding other parameters fixed); the range in axis ratios allow for prolate, oblate, and mixed triaxiality; and $\phi$ is restricted to $\pm45^\circ$ around the value measured by LM10 ($\phi = 97^\circ$). Figure~\ref{fig:potential} shows equipotential slices in the Galactic X-Z plane (Y=0) and Y-Z plane (X=0) for a few choices of  $q_1$,  $q_z$, and $\phi$ while holding all other parameters fixed, as described in the figure caption.

\begin{table}[h]
\begin{center}
	\begin{tabular}{l c c c} \toprule
		{\bf Component} & {\bf Parameter} & {\bf Value} & {\bf Prior} \\\toprule
		disk & $M_{\rm disk}$ & $1.0\times10^{11}M_\odot$ & -- \\ 
		& $a$ & 6.5 kpc & --\\
		& $b$ & 0.26 kpc & --\\
		\midrule
		spheroid & $M_{\rm spher}$ & $3.4\times10^{10}M_\odot$ & --\\ 
		& $c$ & 0.7 kpc & --\\
		\midrule
		halo & $v_h$ & -- & $\mathcal{U}(100,200)$ km $\mathrm{s}^{-1}$ \\
		& $q_1$ & -- & $\mathcal{U}(1,2)$\\
		& $q_2$ & 1.0 & --\\
		& $q_z$ & -- & $\mathcal{U}(1,2)$\\
		& $\phi$ & -- & $\mathcal{U}(52,142)$ deg\\
		& $r_h$ & 12 kpc & --\\
		\bottomrule
		\hline
	\end{tabular}
	\caption{Parameter values used ...\label{tbl:potential}}
\end{center}
\end{table}

\subsection{\gaia + \spitzer}
4 stars in Sgr-like stream

\subsection{\spitzer + current data}
4 stars in Sgr-like stream

\subsection{\spitzer + current data, no progenitor}
4 stars in Orphan-like stream, XX stars in GD-1 -like stream

\section{Discussion}

[Cold vs. Hot streams]

[bow-tie structure, Kuepper work, and so on]

[What's unrealistic about what we have done?  The kinematic model of
  the progenitor is exceedingly simplistic; why are we okay with it
  nonetheless but what could we do to improve it?  Related to this,
  the progenitor doesn't shrink with time (as it should).  What's up
  with that?]

[We did not assume the potential is static in time.  At least not in
  general; nothing in the method requires or suggests this.]

[Issues about computation and scaling.]

[Nothing special about streams; will work on any structures, including
  shells and the like.]

[Don't eschew low-dimensional constraints on the potential!]

[Another prediction of hierarchical structure formation in $\Lambda$CDM is the presence of substructure. In principle, this method would work on any potential that we can integrate orbits in.]

\bibliographystyle{apj}
\bibliography{refs}

\acknowledgements
APW: NSF GRFP, Sigma Xi

\end{document}
