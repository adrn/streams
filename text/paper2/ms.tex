\documentclass[letterpaper,12pt,preprint]{aastex}

% packages
\usepackage{amssymb,amsmath, amsbsy}

\newcommand{\project}[1]{\textsl{#1}}
\newcommand{\gaia}{\project{Gaia}~}
\newcommand{\spitzer}{\project{Spitzer}~}
\newcommand{\given}{\,|\,}
\newcommand{\dd}{\mathrm{d}}
\newcommand{\transpose}[1]{{#1}^{\mathsf{T}}}
\newcommand{\inverse}[1]{{#1}^{-1}}

% bold X and D
\newcommand{\D}{{\bf D}}
\newcommand{\X}{{\bf X}}
\newcommand{\bSigma}{{\bf \Sigma}}
\newcommand{\bsigma}{\boldsymbol\sigma}

\begin{document}

\title{Inferring the gravitational potential of the Milky Way with few, precisely measured stars}
\author{Adrian M. Price-Whelan, DWH, KVJ, others?}

\begin{abstract}
%Informative phase-space structures, and in particular those created by
%disrupting stellar systems or revealed by chemical tags, may deliver
%very precise measures of the gravitational potential in the Milky Way
%Halo.  Here we show that even a single pair of stars---two stars that
%are known (for some non-kinematic reason) to be likely to be
%associated with one another at birth---could provide a significant
%constraint on the potential.  The inference is based on a
%probabilistic generative model---a likelihood function and priors over
%nuisance parameters---that evolves a past putative origin disruption
%event forward in time to the present day.  The time, six-volume, and
%phase-space location of the origin event are nuisance parameters in
%the model and marginalized away.  The method makes no assumption of
%integrability, works with finite or even large observational
%uncertainties, does not require all dimensions of phase space to be
%observed, handles non-zero probability that the two stars are not in
%fact associated, and generalizes naturally to larger numbers of stars
%and multiple independent structures.  Applications to the GD-1 cold
%stellar stream and current surveys for RR Lyrae stars in the Halo are
%discussed.
TODO...
\end{abstract}

\keywords{
  Galaxy: structure
  ---
  Galaxy: halo
  ---
  cosmology: dark matter
}

\section{Introduction}

[Cold streams contain tons of information.]

[GD-1 was fit as if it highlighted an orbit.  That is known to be
  wrong.]

[In the halo, any phase-space structure might be long-lived.  Chemical
  tagging could in principle illuminate it.]

[Chemical tagging will always only give probabilistic information.
  But what if it delivers somewhat confident information about small
  stellar families in the halo?]

[Actions, angles, integrable, not, etc.  Can integrate in any
  potential, even a time-varying one. We use this to
  ``run the clock back''.  Why is an objective based on past collisions not the
  same as assuming that the stars are on the same orbit?]

[Sometimes stellar samples might get extremely small.  For example,
  \gaia\ might find some very low-mass cold stellar streams.  For
  another, there might only be a coupld RRL stars in a particular
  stream, but those RRL stars might be individually extremely well
  measured in six-space.]

[Obviously a small stellar sample can only tell you a small number of
  things about the Milky Way potential.  The idea is---in the long
  run---to construct likelihood functions for many small stellar
  samples and in the end do inference by multiplying them together.
  Duh!]
  
Things to address:
- Density is hard to match -- need background model, way to go from luminosity to density, needs assumptions about the mass function and structure of the satellite
-

\section{The model}

Take ${\bf D}^{(i)}$ to be the observed heliocentric 6D position of the $i$th star -- e.g., the measured $l$, $b$, $D$, $\mu_l$, $\mu_b$, $v_r$ -- and ${\bf X}^{(i)}$ to be the corresponding \emph{true} 6D position of the same star. We assume that each star was once part of a progenitor system---e.g., satellite galaxy---presently at observed position ${\bf D}_s$ (with true position ${\bf X}_s$) that is tidally disrupting and forming a cold debris structure in the potential, $\Phi$, of some parent galaxy. We then model the measured position of a star today as the result of sampling from a time-dependent 6D distribution -- the progenitor system -- at some time $\tau^{(i)}$ in the past, and evolving that sampled 6D position to present day in the parent potential, neglecting the effect of self-interaction with the halo of the progenitor. We want to perform inference \emph{without} building and running---within an MCMC chain---a fully dynamical model of a disrupting satellite in the Milky Way. We therefore approximate the progenitor system as a 6D Gaussian with mean vector given by the position of the satellite at any given time, ${\bf X}_s(t)$, and with shape defined by a time-dependent covariance matrix $\bSigma_s(t)$ (the subscript $s$ refers to satellite or progenitor). With these assumptions, the generative model for the data is written:
\begin{align}
	p(\Phi, \X^{(i)}, \X_s, \bSigma_s, \tau^{(i)} \given \D^{(i)}, \D_s) &\propto 
		p(\D^{(i)} \given \X^{(i)}) p(\D_s \given \X_s) p(\X^{(i)} \given \X_s, \bSigma_s, \tau^{(i)})
\end{align}

Taking $\mathcal{N}(x \given \mu, \Sigma)$ to be the normal distribution over x with mean $\mu$ and covariance tensor $\Sigma$, the individual components are simply:
\begin{align}
	p(\D^{(i)} \given \X^{(i)}) &= \mathcal{N}(\X^{(i)} \given \D^{(i)}, \bsigma^{(i)})\\
	p(\D_s \given \X_s) &= \mathcal{N}(\X_s \given \D_s, \bsigma_s)\\
	p(\X^{(i)} \given \X_s, \bSigma_s, \tau^{(i)}) &= \mathcal{N}(\widetilde{\X}^{(i)}(\tau^{(i)}) \given \widetilde{\X}_s(\tau^{(i)}), \bSigma_s(\tau^{(i)}))\label{eq:likelihood}
\end{align}
where the vectors $\bsigma^{(i)}$ and $\bsigma_s$ represent the (assumed) Gaussian, uncorrelated observational errors on the observed 6D position of star $i$ and the satellite, respectively, and the tildes denote a conversion from spherical, heliocentric to cartesian, Galactocentric coordinates. The Gaussian in Eq.~\ref{eq:likelihood} is evaluated at the time at which the $i$th star becomes unbound, $\tau^{(i)}$.

We first consider the case of a progenitor described only by a single (time-dependent) spatial scale---the tidal radius, $r_{\mathrm{tide}}(t)$---and a single velocity scale---the velocity dispersion, $\sigma_\mathrm{v}$---with \emph{perfectly} observed position and velocity components. Under these assumptions, the posterior is greatly simplified:
\begin{align}
	p(\Phi, \X^{(i)}, \tau^{(i)} \given \D^{(i)}) &\propto 
		p(\D^{(i)} \given \X^{(i)}) p(\X^{(i)} \given \X_s, r_{\mathrm{tide}}, \sigma_\mathrm{v}, \tau^{(i)})\label{eq:posterior}
\end{align}
and thus the $\ln$-posterior is
\begin{align}
	\ln p(\Phi, \X^{(i)}, \tau^{(i)} \given \D^{(i)}) &= \ln p(\D^{(i)} \given \X^{(i)}) + \ln p(\X^{(i)} \given \X_s, r_{\mathrm{tide}}, \sigma_\mathrm{v}, \tau^{(i)}).
\end{align}
The first term on the RHS -- from the observed positions of the stars	-- can be expanded to:
\begin{align}
	\ln p(\D^{(i)} \given \X^{(i)}) &= K - \frac{1}{2}\sum_j \left[ 2\ln\sigma_j + \left(\frac{X^{(i)}_j-D^{(i)}_j}{\sigma^{(i)}_j}\right)^2\right]
\end{align}
where $K$ is a constant related to the dimensionality. 
 
The second term -- quantifying how well the potential ``returns'' the stars to the progenitor -- reads as follows (for the case of a spherical progenitor):
\begin{align}
	\ln p(\X^{(i)} \given \X_s, \bSigma_s, \tau^{(i)}) &= M - \frac{1}{2}\ln(\det \bSigma_s(\tau^{(i)})) - \frac{1}{2}\langle \widetilde{\X}-\widetilde{\X}_s, \bSigma_s^{-1}(\widetilde{\X}-\widetilde{\X}_s) \rangle\\
	&= M - \left[3(\ln r_{\rm tide} + \ln \sigma_v) - \frac{1}{2}\left( \frac{({\boldsymbol r}^{(i)}-{\boldsymbol r}_s)^2}{r_{\rm tide}^2} + \frac{ ({\boldsymbol v}^{(i)}-{\boldsymbol v}_s)^2}{\sigma_v^2} \right)\right]_{t=\tau^{(i)}}\label{eq:ln_like}
\end{align}
where $\boldsymbol r$,$\boldsymbol v$ are the (cartesian) positions and velocities integrated \emph{backwards} in potential $\Phi$ from their present-day positions $\widetilde{\X}$, and $M$ is another additive constant. The whole expression Eq.~\ref{eq:ln_like} is evaluated at the time unbound, $\tau^{(i)}$.

%The marginalization integrals require prior pdfs for the
%parameters---or at least the nuisance parameters.  Here the nuisance
%parameters include the 6-vector position $X_k^{(0)}$ of the
%progenitor, the 6-tensor size $\Sigma_k$ of the progenitor, and all
%the times $t_{nk}$:

Eq.~\ref{eq:posterior} does not include the prior distributions needed to perform marginalizations (integrals) over nuisance parameters of the model. In the most naive model, the unbinding times, $\tau^{(i)}$, could be uniformly drawn from the lifetime of the progenitor; that is, the times could be drawn from a uniform pdf extending from some time, $T$, several Gyr in the past to the present day $t=0$. In what follows we adopt for simplicity the flat prior $p(\tau)=U(\tau\given T, 0)$, where $U(x\given a,b)$ is the uniform distribution for $x$ between limits $a<x<b$.

%A more sophisticated model might have the
%release times be more likely to occur near pericenter passages; that
%is, if the progenitor has finite eccentricity (as it surely will) then
%the probability of releasing a star is probably a strong function of
%Galactocentric distance. 

[Question for Hogg: I don't understand this paragraph -- can't we just ignore a prior on the true star positions?]
The prior on the phase-space position $X_k$ could be built from some
kind of maximum-entropy considerations in some baseline potential or
in the current setting of the model potential parameters.  Or it could
be generated from empirical distributions found for known substructures or
satellites found in data or simulations of galaxy formation.  For
simplicity we choose the 3-vector position part of $X_k$ to be drawn
from the singular isothermal sphere $p(x)\propto |x|^{-2}$ in the
range [HOGG]$<|x|<$[HOGG]~kpc and the 3-vector velocity part to be
drawn from a spherical (diagonal) Gaussian of velocity dispersion
[HOGG].  These priors are not really justified by prior knowledge, but
they also don't matter enormously when the data are informative.  When
we test the method (below), we will explicitly test in cases where
these priors are inappropriate, to demonstrate that they don't
substantially affect the results in realistic situations.

The prior on the progenitor 6-tensor size $\Sigma_k$---which really
has only two degrees of freedom, $\sigma_{kx}^2$ and $\sigma_{kv}^2$, the
diagonal elements of the 3-tensor spatial extent and the diagonal
elements of the 3-tensor velocity extent---must enforce positivity and
(relatively speaking) ``coldness'' of the structure.  We relatively
arbitrarily put [HOGG] priors on these with parameters [HOGG].

[Milky Way ``background'' model.  Mixture model.]

[Generalization to multiple phase-space structures. Mixture of
  mixtures.]

[Comment on bow-tie structure, Kuepper work, and so on;
  might want to generalize the Gaussian brutality.]

\section{Tests on simulated data}

[Each experiment requires a description of how the fake data were
  generated, what the likelihood function is, and posterior potential
  information obtained.]

\subsection{\gaia + \spitzer}

\subsection{\spitzer + current data}

\section{Discussion}

[What's unrealistic about what we have done?  The kinematic model of
  the progenitor is exceedingly simplistic; why are we okay with it
  nonetheless but what could we do to improve it?  Related to this,
  the progenitor doesn't shrink with time (as it should).  What's up
  with that?]

[We did not assume the potential is static in time.  At least not in
  general; nothing in the method requires or suggests this.]

[Issues about computation and scaling.]

[Don't be afraid of nuisance parameters!]

[Nothing special about streams; will work on any structures, including
  shells and the like.]

[Don't eschew low-dimensional constraints on the potential!]

\acknowledgements
Binney, Rix, Sanders

\end{document}
