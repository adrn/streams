%\documentclass{emulateapj}
\documentclass[letterpaper,12pt,preprint]{aastex}

% packages
\usepackage{amssymb,amsmath, amsbsy}
\usepackage{booktabs}

% commands
\newcommand{\project}[1]{\textsl{#1}}
\newcommand{\given}{\,|\,}
\newcommand{\dd}{\mathrm{d}}
\newcommand{\transpose}[1]{{#1}^{\mathsf{T}}}
\newcommand{\inverse}[1]{{#1}^{-1}}
\newcommand{\jac}[1]{\left\vert \J{#1} \right\vert}

\newcommand{\rtide}{r_{\rm tide}}
\newcommand{\bs}{\boldsymbol}

\begin{document}

\title{The Phase-space Properties of Tidal Debris}
\author{Adrian M. Price-Whelan\altaffilmark{\colum,\adrn}, 
	    Kathryn V. Johnston\altaffilmark{\colum}, 
	    David Hendel\altaffilmark{\colum}}

% Affiliations
\newcommand{\colum}{1}
\newcommand{\adrn}{2}
\altaffiltext{\colum}{Department of Astronomy, 
		              Columbia University, 
		              550 W 120th St., 
		              New York, NY 10027, USA}
\altaffiltext{\adrn}{To whom correspondence should be addressed: adrn@astro.columbia.edu}

\begin{abstract}
% Context
The tidal disruption of collisionless systems of particles forms observable debris structures in galaxies. Cold debris structures are excellent probes of the underlying potential or mass distribution of galaxies and are detectable over a wide range of distances from the inner [...] out to the virial radius. 
% Aims
We show that the properties of the debris can be thought of as generated from two distinct processes --- tidal leaking and tidal shocking --- and the extent of the debris scales in energy (or phase-space) with the mass-ratio between the system and the parent potential.
% Methods: 
We run N-body simulations of satellite galaxies and globular clusters on two qualitatively distinct orbits [...]
% Results:
% Conclusions: 
\end{abstract}

\keywords{}

\section{Introduction}

\section{Scalings}
The effective potential for the three-body problem for two point masses

\section{Simulations}
[David Hendel]
We [...] N-body simulations run with an SCF code [...]. [describe simulations here] [Describe tub]

\end{document}