\documentclass[preprint]{aastex}

\begin{document}
\title{Probing the Milky Way's Dark Matter Halo with RR Lyrae Stars} \author{Adrian M.
  Price-Whelan\altaffilmark{1}, Kathryn Johnston\altaffilmark{1},
  Allyson Sheffield\altaffilmark{1}} \altaffiltext{1}{Department of
  Astronomy, Columbia University, Mail Code 5246, New York, NY 10027}

Galaxies, by mass, are mostly composed of \emph{Dark
  Matter}. Simulations of structure and galaxy formation -- both with
and without baryons -- lead us to believe that galaxies live in halos of
dark matter that have universal density profiles \citep{nfw96} and are
triaxial in shape \citep{jing02}. Including baryons in such
simulations tends to soften the triaxiality and alters the radial
density profile of the dark matter. Thus, precise measurements of the 
shape, orientation, and radial profile of dark matter halos provide 
constraints on both dark matter physics and the baryonic processes 
that helped shape them. 

Tidal streams and substructure in the Milky Way's 
halo (from disrupted, infalling satellite galaxies) have been used
to constrain properties of the mass distribution around the Milky Way \citep[e.g.,][]{law10}.
Typically found outwards of 20~kpc from the Galactic center, the orbits 
of associated stars are still largely under the influence of the dark matter 
and thus serve as tracers of mass \citep[e.g.,][]{johnston99, pricewhelan13}. 
Tens of stellar 
streams and spatial overdensities have been identified in the Sloan
Digital Sky Survey (SDSS) and Two-Micron All Sky Survey (2MASS)
and a variety of other deep, photometric studies \citep[e.g.,][]{majewski03,belokurov06}. 
For most of these interesting relics of the Milky Way's merger history, 
only sky positions and rough photometric distances have been measured. 
A combination of better distance measurements with follow-up spectroscopy
would allow for a more detailed reconstruction of these stellar overdensities 
by enabling direct measurements of radial velocities and metallicities --- 
both powerful tools for distinguishing populations of stars formed in satellite 
galaxies from the background of Milky Way halo stars.

Measuring distances to stars is notoriously difficult. Recently, \citet{madore12}
have shown that a class of variable star --- RR Lyrae stars --- have a 
tight period-luminosity relation in the mid-infrared. By comparing distances
measured using this relation to five stars stars with trigonometric parallaxes 
measured by Hubble, they demonstrate that it is possible to use observations with
the Spitzer space telescope to determine distances that are good to 2\% for individual RR Lyrae 
stars out to $\sim$60~kpc. As a part of a much
larger collaboration, we have applied for observing time with Spitzer to measure
mid-infrared light curves for hundreds of stars in the Sagittarius stream \citep[Sgr;][]{majewski03},
the Orphan stream \citep[Orp;][]{belokurov07}, and the Triangulum-Andromeda 
overdensity \citep[TriAnd;][]{rochapinto04}.

We have already received time on the 2.4m Hiltner telescope at MDM 
observatory (August 2013 and October 2013) to obtain spectra for stars
associated with TriAnd. We are applying for i) support for travel for three
observing runs to be conducted this Spring (2014) to follow up both 
calibration stars and Sgr and Orp stream stars targeted with Spitzer, and ii) 
equipment for data reduction and storage.
Combined with future data (proper motions) from the ESA's Gaia mission,
we will for the first time have precise, 6D kinematic information
for samples of stars at large distances in the Galactic halo. This data will 
enable us to model the large-scale mass distribution around the Milky Way 
with unprecedented detail.

\begin{thebibliography}{ }
\bibitem[{{Belokurov} {et~al.}(2006){Belokurov}, {Zucker}, {Evans}, {Gilmore},
  {Vidrih}, {Bramich}, {Newberg}, {Wyse}, {Irwin}, {Fellhauer}, {Hewett},
  {Walton}, {Wilkinson}, {Cole}, {Yanny}, {Rockosi}, {Beers}, {Bell},
  {Brinkmann}, {Ivezi{\'c}}, \& {Lupton}}]{belokurov06}
{Belokurov}, V., {Zucker}, D.~B., {Evans}, N.~W., {et~al.} 2006, \apjl, 642,
  L137
\bibitem[{{Belokurov} {et~al.}(2007){Belokurov}, {Evans}, {Irwin},
  {Lynden-Bell}, {Yanny}, {Vidrih}, {Gilmore}, {Seabroke}, {Zucker},
  {Wilkinson}, {Hewett}, {Bramich}, {Fellhauer}, {Newberg}, {Wyse}, {Beers},
  {Bell}, {Barentine}, {Brinkmann}, {Cole}, {Pan}, \& {York}}]{belokurov07}
{Belokurov}, V., {Evans}, N.~W., {Irwin}, M.~J., {et~al.} 2007, \apj, 658, 337
\bibitem[Jing \& Suto(2002)]{jing02} Jing, Y.~P., \& Suto, Y.\ 2002, \apj, 574, 538
\bibitem[Johnston et al.(1999)]{johnston99} Johnston, K.~V., Zhao, H., Spergel, D.~N., \& Hernquist, L.\ 1999, \apjl, 512, L109 
\bibitem[Law \& Majewski(2010)]{law10} Law, D.~R., \& Majewski, S.~R.\ 2010, \apj, 714, 229 
\bibitem[Madore \& Freedman(2012)]{madore12} Madore, B.~F., \& Freedman, W.~L.\ 2012, \apj, 744, 132 
\bibitem[{{Majewski} {et~al.}(2003){Majewski}, {Skrutskie}, {Weinberg}, \&
  {Ostheimer}}]{majewski03}
{Majewski}, S.~R., {Skrutskie}, M.~F., {Weinberg}, M.~D., \& {Ostheimer}, J.~C.
  2003, \apj, 599, 1082
\bibitem[Navarro et al.(1996)]{nfw96} Navarro, J.~F., Frenk, C.~S., \& White, S.~D.~M.\ 1996, \apj, 462, 563
\bibitem[Price-Whelan \& Johnston(2013)]{pricewhelan13} {Price-Whelan}, A.~M. \& {Johnston}, K.~V. {arXiv:1308.2670}, August 2013.
\bibitem[{{Rocha-Pinto} {et~al.}(2004){Rocha-Pinto}, {Majewski}, {Skrutskie},
  {Crane}, \& {Patterson}}]{rochapinto04}
{Rocha-Pinto}, H.~J., {Majewski}, S.~R., {Skrutskie}, M.~F., {Crane}, J.~D., \&
  {Patterson}, R.~J. 2004, \apj, 615, 732
\end{thebibliography}

\end{document}
