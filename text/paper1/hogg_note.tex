\documentclass[12pt,letterpaper]{article}

\newcommand{\TT}{{\!\mathsf{T}}}

\begin{document}

\section*{to Adrian from Hogg}

Right now, I believe that you do \emph{something} like:
\begin{blockquote}
For each star,
evolve it's phase-space position $X$ (a 6-vector) backwards in time $t$.
Find the time $t_0$ at which it's 6-space distance $\Delta_0$ from the progenitor is minimized.
Compute the empirical variance tensor $V$ by taking the mean outer product $\Delta_0\cdot \Delta_0^\TT$.
Use $\ln\det V$ as the scalar objective by which different potentials are compared.
\end{blockquote}
I am not sure if this summary is correct.
For example, in the code snippet you tweeted, you are subtracting a mean from the 6-displacements.
That implies that perhaps you don't know where the progenitor really is?
But if so, then how do you find the time $t_0$ at which 6-displacement is minimized?

Here's my proposal:

\end{document}
