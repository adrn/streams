%\documentclass{emulateapj}
\documentclass[preprint]{aastex}
%\documentclass{aastex}
%\usepackage{emulateapj5}
%\usepackage{lscape}
\usepackage{natbib}
\usepackage{amssymb,amsmath}
\usepackage{hyperref}

% Convenience
\newcommand{\bs}{\boldsymbol}
\newcommand{\argmin}{\operatornamewithlimits{argmin}}

\setlength{\jot}{12pt}

\begin{document}

\title{Spitzer, Gaia and the Potential of the Milky Way}

\author{Adrian~M.~Price-Whelan\altaffilmark{1,2} Kathryn V. Johnston\altaffilmark{1}}

% affiliations
\altaffiltext{1}{Department of Astronomy, Columbia University, 550 W 120th St., New York, NY 10027, USA}
\altaffiltext{2}{NSF Graduate Fellow}

\begin{abstract}
.
\end{abstract}


\section{Introduction}
\label{intro.sec}
 The existence of vast halos of unseen {\it dark} matter surrounding each galaxy has long been proposed \citep[e.g.,][]{zwicky37,rubin70} to explain the surprisingly large
motions of the {\it baryonic} matter (i.e. stars, gas and dust) that we can see.
Dark-matter-only simulations of structure formation lead us to expect that these  halos should have a density distribution that can simply be described with a universal
profile \citep{nfw96} with a variety of triaxial shapes \citep{jing02}.
The inclusion of baryons in the simulations tends to soften the triaxiality of the dark matter in the inner regions of the halo \citep[as the disk forms, see e.g.,][]{bailin05}, and
can potentially alter the radial profile through a combination of adiabatic contraction and energetic feedback \citep[see recent discussion by][]{pontzen12}.
Hence, measurements of the shape, orientation, radial profile and extent of dark matter halos can shed light both on the formation of these vast structures and the role that
the messy baryonic processes of gas inflow, star formation and feedback  have played in shaping them.

The one place in the Universe where we might hope to dissect a dark matter halo in even more detail is locally for our own Milky Way Galaxy.
Our proximity means that we can use individual stars as tracer populations and hence build much larger samples that probe deeper into the halo than the globular cluster
and planetary nebula studies of external galaxies.
Moreover, our internal perspective allows us to build a three-dimensional picture of the mass distribution rather than measuring its two-dimensional projection, as is the
case for gravitational lensing.
Hence we can hope to reconstruct not only the density, but also the shape and orientation of the dark matter halo as function of radius well beyond the disk.
For example, to date thousands of blue-horizontal-branch stars selected from the Sloan Digital Sky Survey \citep[SDSS][]{} have been used to probe mass out to tens of
kpc \citep{deason12,kafle12}, and estimates with combined tracers extend to 150kpc \citep{deason12b}.
\citet{loebman12} demonstrates how even larger stellar samples from current (i.e. SDSS) and future  (Gaia and LSST) stellar surveys will also be sensitive to any
asphericity of the halo.

There is a caveat to this work using tracer populations: these studies all rely on the assumptions that the tracers  represent a random sampling of orbits drawn from a
smooth distribution function and are well-mixed in orbital phase.
However, the same surveys that provide large samples of tracers have also revealed the existence of large-scale spatial inhomogeneities in the form of giant stellar
streams and clouds\citep{newberg01,majewski03,belokurov05}, clearly demonstrating that  a significant fraction of the stellar halo is neither randomly sampling a smooth
orbit distribution nor is fully phase-mixed.

A complimentary method to measure the mass distribution is to instead take full advantage of the {\it non}-random nature of the stellar distribution in our stellar halo ---
analogous to our exploitation of matter in disks moving on near-circular orbits.
While we do not know the exact orbits of  stellar debris from satellite destruction \citep[although][proposes a method of orbit reconstruction from limited dimensions of
observations]{eyre08}, we do know something more about stars in a debris structure than in a random sample: we know that these stars were once all part of the same
object.
This approach is very promising --- because tidal streams are dynamically cold systems, it requires orders of magnitude fewer tracers than a random sample to get
constraints of comparable accuracy.
In the simplest case, we might {\it assume} that debris stars are actually still on the same orbit as their progenitor system.
Then we can imagine that measuring the full-space velocities ${\bf v}$ at different points ${\bf x}$ in the structure (e.g. along a stream) would actually give us a direct
measure of differences in potential $\Phi$ (i.e. $\Phi({\bf x}_1)-\Phi({\bf x}_2)=(v_2^2-v_1^2)/2$).
In reality, we usually know at most four of the six phase-space co-ordinates for debris stars, often with significant error bars and potential measurements are made by
fitting orbits to streams \citep[e.g.,][]{helmi05,johnston05,koposov10,law10,lux12}.
Moreover, our assumption of debris tracing a single orbit is actually incorrect \citep[see][for discussions of the orbit distribution in tidal debris]{johnston98,helmi99}, with
systematic changes in orbital energy along debris streams that can lead to systematic biases in our measurements of the Galactic potential \citep{eyre09,varghese11}.

One way to address these systematic biases is to run fully self-consistent N-body simulations of satellite destruction in a variety of potentials (which naturally generate a
distribution of orbital properties in the debris and along tidal streams) with the aim of simultaneously constraining both the properties of the satellite and the Milky Way.
Many studies of the Sagittarius debris system (hereafter Sgr)  --- whose tidal streams entirely encircle the Milky Way --- have adopted this approach \citep[e.g.][]
{law05,fellhauer06} with the most recent work for the first time attempting to place constraints on the triaxiality and orientation of the Milky Way's matter distribution
\citep{law11}.

Other approaches have been proposed that avoid the cost of full N-body simulations yet recognize the limitations of single orbit integration.
For example, using the results of N-body simulations as a guide, simple corrections around a single orbit can be adopted to make model predictions that more closely
mimic expected debris behavior \citep[e.g.][]{johnston99a,varghese2011}.

The promise of near-future data sets including full (or nearly full) phase-space information has also inspired discussions that move beyond fitting observables along a
stream.
\citet{binney08} and \citet{penarrubia12} demonstrate conceptually that the distribution of energy and entropy (respectively) in debris will be minimized only for a correct
assumption of the form of the Galactic potential (although neither tests their method with realistic observational errors???).

In this paper we re-examine and update  the JZSH  in the context of current and near-future observational capabilities, and illustrates its power with the example of Sgr.
In Section \ref{context.sec} we outline the observational prospects and Sgr properties that motivated this re-examination.
In Section \ref{measure.sec} we present the updated potential measure and test it with synthetic observations of simulated Sgr debris.
In Section \ref{summary.sec} we summarize our results and outline prospects for combining constraints from Sgr with those from other debris systems.


% --------------------------------------------------------------------
% Above here is from KVJ. Below is APW.
%

\section{Context and motivation}
The method presented in Section \ref{sec:method} takes advantage of
three distinct developments in the observations of tidal streams: (i)
the demonstration of a technique for deriving distances to individual
RR Lyrae stars with 2\% accuracies (Section \ref{sec:spitzer}); (ii)
the prospect of proper motion measurements of the same stars with
$\sim$10~$\mu$as/yr precision (Section \ref{sec:gaia}); and (iii) the
tracing of debris associated with Sgr around the entire Galaxy
(Section \ref{sec:sgr})

\subsection{2\% distance errors for RR Lyrae in the halo}
\label{sec:spitzer}
Over twenty-five years ago, \citet{longmore1986} demonstrated the
existence of a near-infrared period-luminosity (PL) relation for RR
Lyrae stars using $K$-band observations of stars in Galactic
globular clusters. More recently, \citet{catelan2004} showed that a PL
relation for RR Lyrae stars is expected in the infrared (but
vanishingly so in the optical) and is explained primarily by the
changing bolometric correction for these stars as a function of the
observed wavelength. Later, \citet{madore2012} demonstrated the
inevitability of decreasing intrinsic scatter with increasing
wavelength (and slope) of PL relations both for Cepheids and for RR
Lyrae stars. Both reddening effects and metallicity dependencies also
decrease significantly for observations further into the IR.

These studies suggest that observations of RR Lyrae variables by
NASA's Spitzer mission could provide us with a stellar distance
indicator of unprecedented accuracy and precision. \citet{madore2012}
have demonstrated \citep[using five stars with trigonometric parallaxes
  measured by Hubble;][]{benedict2011} that the dispersion in the
mid-IR PL relation -- at Spitzer wavelengths -- is of order
$\sim$0.03~mag. This implies that it is possible to determine
individual distances that are good to better than $2\%$ for RR
Lyrae stars out to $\sim$80~kpc (the limit of Spitzer's ability to
detect and measure RR Lyrae variables, for modest integration
times). Distance measurements of Blue Horizontal Branch and Blue Straggler stars at these distances in the halo typically achieve $\sim$25\% uncertainties \citep[e.g.,][]{deason2012}. 

\subsection{Gaia and the age of astrometry}
\label{sec:gaia}
The Gaia satellite \citep{gaia2001}, planned for launch in late
2013, is an astrometric mission which aims to measure the positions of billions of stars with
10-100~$\mu$as accuracies. With over 50 observations of the entire sky
during the 5-year lifetime of the mission, Gaia will
revolutionize the field of Galactic Astronomy by providing sufficient
proper motion accuracy to create full six-dimensional phase-space
maps of the Galaxy with $<$10\% distance errors for heliocentric distances up to $\sim$6~kpc for RR Lyrae stars.

Figure~\ref{fig:gaia_errors} show the Gaia end-of-mission distance and
tangential velocity error estimates for RR Lyrae. The velocity uncertainty depends on the magnitude of the source, which depends on the metallicity: each line in Figure~\ref{fig:gaia_errors} is computed by Monte Carlo sampling from the empirical metallicity distribution of the Galactic halo from \cite{ivezic2008} (centered around [Fe/H] $\sim$ -1.5). Within 2~kpc, Gaia
will be able to measure distances to these stars with better than 2\%
accuracy---RR Lyraes in this volume will be used to test and calibrate
the Spitzer PL relation described above. Beyond the 2~kpc threshold,
the mid-IR PL relation for RR Lyraes will provide more accurate
distance measurements.

\begin{figure}[h]
\begin{center}
\includegraphics[width=1.\textwidth]{/Users/adrian/projects/streams/plots/paper1/gaia.pdf}
\caption{ todo }\label{fig:gaia_errors}
\end{center}
\end{figure}

The anticipated proper motion errors for RR Lyrae correspond to
tangential velocity errors of order $\sim$1~km/s at 30~kpc, and of
order $\sim$10~km/s out to several tens of kiloparsecs. The
combination of Spitzer and Gaia data will extend the `horizon' of
where full six-dimensional phase-space maps of the Galaxy are possible
from $<$10~kpc to 80~kpc. This enormous increase in volume will greatly refine
data on existing debris systems such as the Sagittarius stream.

\subsection{The Sagittarius debris system}
\label{sec:sgr}
The Sagittarius Dwarf Galaxy (hereafter Sgr) was discovered
serendipitously during a radial velocity survey of the Galactic bulge
\citep{ibata1994}. While the number of known Milky Way satellites has
more than doubled recently \citep[e.g.,][]{mcconnachie2012}, in many
ways Sgr remains the most studied and most intriguing: it is the
closest satellite, among the largest (only the LMC and SMC are more
luminous), and has the most prominent tidal tails.

The first maps of the Sgr core \citep{ibata1994, ibata1995} revealed
its elongated morphology and suggested tidal deformation, as would be
anticipated from its proximity to the Galaxy. Subsequent N-body models
confirmed this interpretation \citep{velazquez1995} and showed that
debris from Sgr should form coherent streams of stars that encircle
the Galaxy \citep{johnston1995}. Signatures of these extensive stellar
streams have since been discovered in many studies and have been
mapped across the sky in carbon stars \citep{totten1998}, M giants
selected from the Two Micron All Sky Survey \citep{majewski2003}, main
sequence turnoff stars from the Sloan Digital Sky Survey
\citep{belokurov2006}, and RR Lyrae in the Catalina Sky Survey
\citep{drake2013}.

The distance to the stream has been estimated from photometric surveys
\citep[e.g.,][]{martinezdelgado2004} and gradients in radial
velocities and metallicity measured with follow-up spectroscopy
\citep[e.g.,][]{majewski2004, vivas2005, bellazzini2006, chou2007,
  chou2010, keller2010, carlin2012}. The multitude of Sgr stream data
has inspired a rich set of interpretations and models of the debris
\citep[e.g.,][]{johnston1999, helmi2004, law2005, fellhauer2006,
  law2010}.

Most recently, \citet[][hereafter LM10]{law2010} have combined all
current data on the Sgr debris to assess the triaxiality and orientation
of the outer Galaxy, the first time that such a reconstruction of
the 3-dimensional mass distribution of a dark matter halo has been
feasible. Figure \ref{fig:lm10} shows particle positions from the final time-step of an N-body simulation of dwarf satellite disruption along the expected Sgr
orbit in the best-fitting Milky Way halo model from LM10. [TODO: Maybe not true...] The positions are shown in a coordinate system defined
by the orbit of the satellite \citep{majewski2004}. The
simulation was run in a three-component potential, with a triaxial,
logarithmic halo model of the form
\begin{equation}
  \Phi_{halo} = v_{halo}^2 \ln(C_1 x^2 + C_2 y^2 + C_3 xy + (z/q_z)^2 + R_c^2)
\end{equation}
where $C_1$, $C_2$, and $C_3$ are combinations of the $x$ and $y$ axis ratios ($q_1$, $q_2$) and orientation of the halo with respect to the baryonic disk ($\phi$):
\begin{align}
  C_1 &= \frac{\cos^2\phi}{q_1^2} + \frac{\sin^2\phi}{q_2^2}\\
  C_2 &= \frac{\sin^2\phi}{q_1^2} + \frac{\cos^2\phi}{q_2^2}\\
  C_3 &= 2\sin\phi\cos\phi \left(q_1^{-2} - q_2^{-2}\right).
\end{align}

LM10 demonstrate how a comparison of such simulations with the
observed debris enables a detailed reconstruction of Sgr's history:  
its mass and stellar populations, orbit, rate of destruction, and
even original distribution in populations [other refs??]. The phase-space distribution of the debris offers a unique probe of
the depth, shape, extent, and orientation of the Milky Way's dark
matter halo that cannot be matched using other techniques on our own
or other galaxies \citep[e.g.,][]{ibata2001}. The LM10 constraints on
Sgr's mass and orbit have invigorated discussions of Sgr in a more
cosmological context, the mass and extent of the original dark matter
halo that hosted it, as well as its effect on the Milky Way itself
\citep{bailin2003,purcell2011,micheldansac2011,gomez2012}.

Combined Spitzer and Gaia measurements of distances and proper
motions of RR Lyraes in the Sgr debris will open up new avenues of
interpretation. A 2\% distance error at $\sim$30~kpc ($\epsilon_{d}=0.6~{\rm kpc}$) is smaller than
the predicted thickness of the stream, ranging from
$\sim$2-$10$~kpc. Combined with Gaia tangential velocity measurement
accuracies ($\epsilon_{v}\sim2$~km/s at 30~kpc) less than the intrinsic dispersion
for much of the stream \citep[$\sigma_v\sim10$~km/s;][]{majewski2004}, and ground-based radial
velocities, the measured phase-space
position of each individual RR Lyrae becomes a
powerful probe of the potential (see \ref{fig:}). The next section outlines a new
method to take advantage of this information.

\section{Description and test of our algorithm}
\label{sec:method}
With access to full phase-space information for stars in a tidal
stream, it is possible to use the individual stars as
potential measures by exploting the fact that the stars must have come
from the same progenitor. If the present-day phase-space position of the
progenitor system is also known, this places an intuitive constraint on
the potential of the Milky Way: if the orbits of the stars and the
orbit of the progenitor are integrated \emph{backwards} in time
(imagine watching a visualization of a simulation of satellite destruction "rewind"), the
stars should `recombine' with the progenitor. If the potential is
incorrect, the orbits of the stars will diverge from that of the
progenitor and thus will not be recaptured by the satellite system (Figure~\ref{fig:ps_distance}).

This approach was originally propose by \citep{jzsh1999} and designed and tested for the proposed characteristics of the Space Interferometry Mission \citep{unwin??}, which was expected to be able to measure proper motions with $\mu$as/year accuracy for stars as faint as $V=20$. Distances to stream members would only be poorly known and were estimated in the method by using anticipated properties of tidal debris. Below we present an updated version of the algorithm: the promise of 2\% distances to RR Lyrae stars (see Section \ref{sec:spitzer}) enables a direct measurement (rather than approximate estimate) of the position of a star within its debris structure; and the test statistic that quantifies how well stars recombine with the satellite has been rigorously redefined.

\begin{figure}[h]
\begin{center}
\includegraphics[width=1.\textwidth]{/Users/adrian/projects/streams/plots/paper1/ps_distance.pdf}
\caption{ todo }\label{fig:ps_distance}
\end{center}
\end{figure}

\subsection{The algorithm: Rewinder}
Quantifying this method requires a sample of stars with known full
space kinematics $(\bs{x}_{i}, \bs{v}_{i})|_{t=0}$ (e.g., measurements of all position and velocity components for these stars \emph{today} at $t=0$), the orbital parameters
for the progenitor system $(\bs{x}_p, \bs{v}_p)|_{t=0}$, and a
functional form for the potential, $\Phi({\boldsymbol\theta})$. For a
given set of potential parameters, $\boldsymbol\theta$, the orbits of the stars and
progenitor are integrated backwards for several Gigayears (determined from an
estimate of the interaction time of the system). The stars and
progenitor system are treated as test particles, so that the debris
stars do not become physically re-bound to the satellite. Instead, at each
timestep $t_j$, for each particle $i$, a set of normalized, relative
phase-space coordinates are
computed 
\begin{equation}
  \bs{q}_{i} = \frac{\bs{x}_{i} -
    \bs{x}_{p}}{R_{\rm tide}}\,\,\,\,;\,\,\,\,\bs{p}_{i} = \frac{\bs{v}_{i} -
    \bs{v}_{p}}{v_{\rm esc}}
\end{equation}
where $(\bs{x},\bs{v})_{i}$ and $(\bs{x},\bs{v})_{p}$ are the phase-space coordinates for the particles and progenitor, respectively. These definitions require an estimate of the mass of the
satellite, $m_{sat}$, which, combined with the computed enclosed mass
of the potential, $m_{enc}$, sets the instantaneous tidal radius and
escape velocity,
\begin{equation}
  R_{\rm tide}=R_{{\rm orbit}}\Big(\frac{m_{\rm sat}}{3m_{\rm enc}}\Big)^{1/3}\,\,\,\,;\,\,\,\,
  v_{\rm esc}=\sqrt{\frac{2Gm_{\rm sat}}{R_{\rm tide}}}.
\end{equation}

Qualitatively, when the distance in this normalized 6D space,
$D_{\rm ps}=\sqrt{|\bs{q}_{i}|^2+|\bs{p}_{i}|^2} < \sqrt{2}$, the star is
probably recaptured by the satellite, but this is not imposed as a
hard constraint. [JZSH ...] As a statistical measure of how many stars have recombined, we
construct the 6D distribution of the minimum phase-space vectors for
each star over each individual, integrated orbit. For each star, $i$,
the phase-space distance, $D_{\rm ps}$, is computed at each timestep
$t_{j}$, and the vector with the minimum phase-space distance is stored
\begin{align}
  t^*_{i} &= \argmin_{t} D_{{\rm ps},i}\\
  \bs{A}_{i} &= (\bs{q}_{i}(t^*_{i}),\bs{p}_{i}(t^*_{i})).
\end{align}
Thus, the matrix $\bs{A}_{ik}$ contains these minimum phase-space distance vectors for each star,
where $k\in\{1,6\}$. Intuitively, the variance of this distribution of minimum distance vectors
will be larger for orbits integrated in an incorrect potential
relative to the distribution computed from the `true' orbital history
of the stars; in an incorrect potential, small differences in the
orbits of the stars relative to the orbit of the progenitor cause them
to spread out in this relative, normalized phase space. Thus, the
\emph{generalized variance} of the distribution, computed for a given
set of potential parameters, $\bs{\theta}$, is a natural choice for
the scalar objective function used in constraining the potential of
the Milky Way,
\begin{equation}
  f(\bs{\theta})=\det(\mathrm{Cov}(A_{ik})).
\end{equation}
[This objective function is an approximation to a more general likelihood function, described in detail in forthcoming work (Price-Whelan \& Johnston in prep.). The results shown in sections xx and yy are meant to illustrate the strength and concept of the method..]

\subsection{Application to Simulated Data}
The LM10 simulation data (see Section \ref{sec:sgr}) is a perfect
test-bed for evaluating the effectiveness of this method. We start by
extracting both particle data and the satellite orbital parameters
from the present-day snapshot of the simulation
data.\footnote{\url{www.astro.virginia.edu/~srm4n/Sgr/data.html}} We then
`observe' a sample of 100 stars from the first leading and trailing
wraps of the stream: the radial velocity errors are drawn from a
Gaussian $\epsilon_{rv} \sim \mathcal{N}(\mu=0,\sigma=10~{\rm km/s})$,
the tangential velocity errors depend on the distance to the star and
are computed from the expected Gaia error curve (shown in
Figure~\ref{fig:gaia}), and the distance errors on a star of distance
$D$ are $\epsilon_{dist} \sim \mathcal{N}(0,0.02\times
D)$. Figure~\ref{fig:} shows all 2D projections of the 6D minimum
phase-space distance distribution for the observed particles in the
correct and an incorrect potential. Qualitatively, it is clear that the
variance of the distribution increases in the wrong potential. The objective function
(generalized variance) defines a convex hypersurface over which we
optimize four of the six logarithmic potential parameters:
$v_{circ}$, $\phi$, $q_1$, and $q_z$. Figure~\ref{fig:} shows [...]
curves produced by varying the true parameters of the potential by
10\% around the true value.

% APW Note: The optimization, in a way, is getting a bunch of orbits (curves) to intersect another curve (the progenitor orbit) at some points (over time).

\section{Discussion and future prospects}

Many orbit fitting methods for measuring the Galactic potential assume that the stars in a given stream are on the same
orbit. \cite{binney??} show that this assumption leads to biased
potential parameters, even without realistic observational errors on
the kinematic data. The method presented above is
free of this assumption and instead \emph{relies} on the fact that
each star is on a different orbit and is thus an independent potentiometer. The power of this method is in
the joint constraint that the best potential is the one that gets the
most of these individual orbits to intersect the progenitor
over the history of interaction with the system.

[merge these paragraphs (above, below)]

The great strength of the  JZSH ({\it can we come up with a catchy name?}) method is its simplicity:
it requires only the assumption a satellite mass $m_{\rm sat}$ and  backwards integration of orbits.
There are no assumptions made about or N-body modeling required to follow the internal distribution of satellite stars or accurately assess the degree of alignment of
debris with a single orbit.
In fact, the method relies on the ability to measure {\it differences} in orbital properties --- these differences result in different orbital time periods for the debris and the
satellite which leads to their paths intersecting during the backwards integration.

There are two other apparent strengths of the method which require further investigation.
First it appears applicable to any debris that is known to come from a single object, and not restricted only to the very coldest tidal streams. For example, in principle the
method could  applied to the vast debris {\it clouds} that have been discovered subtending hundreds-thousands square degrees \citep[e.g. the Triangulum Andromeda
and Hercules-Aquila clouds]{rochapinto04,belokurov06}, or even stars that have only associations in orbital properties and do not form a coherent spatial structure
\citep[such as the angular momentum groupings in local giants found by][]{helmi99}.
Second, it appears trivial to combine constraints from multiple debris systems at once by simply integrating all debris from several satellites simultaneously with $D_{\rm
ps}$ defined appropriately for each star separately.

The limitations of the method also need to be understood. In particular, for the RR Lyraes associated with  the very coldest streams \citep[e.g. the globular clusters Pal5
and GD1][]{odenkirchen02,koposov10} the measurement errors are unlikely to be small enough for the
the differences in orbital properties to between debris and satellite to be detected, and the method should fail.

Lastly, the current version of the algorithm relies on accurate knowledge of the phase-space position of the parent satellite, which may not be available in some cases
\citep[for example, for the Orphan Stream][]{belokurov07}. Given sufficient data, position and motion of the original host could be solved for as additional free parameters,
but the effectiveness of this approach has yet to be investigated.


\section{Summary}

This paper outlines an algorithm to measure the Galactic potential that takes maximal advantage of a combination of possible data from the Spitzer and Gaia satellite
missions that, when combined,  promise full phase-space measurements of RR Lyrae stars around our Galaxy.
When applied to a synthetic data set generated from debris in an N-body simulation of the destruction of Sgr the method recovers the shape and orientation of the dark
matter halo component of the potential with ??? accuracy using a sample of ??? stars.
This success provides strong motivation for: (i) further theoretical work to investigate the power of multiple debris structures with different morphologies and orientations in
building a comprehensive picture of the Galactic mass distribution; and (ii) a  Spitzer survey of RR Lyrae stars in debris structures around the Milky Way as a ``legacy''
sample to combine with near-future Gaia data.

\acknowledgments
We thank ...

APW is supported by a National Science Foundation Graduate Research Fellowship under Grant No.\ 11-44155.

\bibliography{refs}
\bibliographystyle{apj}

\end{document}
