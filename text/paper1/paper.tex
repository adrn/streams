%\documentclass{emulateapj}
\documentclass[preprint]{aastex}

%\documentclass{aastex}
%\usepackage{emulateapj5}
%\usepackage{lscape}
\usepackage{natbib}
\usepackage{amssymb,amsmath}
\usepackage{hyperref}

% Convenience
\newcommand{\bs}{\boldsymbol}
\newcommand{\argmin}{\operatornamewithlimits{argmin}}

\setlength{\jot}{12pt}

\begin{document}

\title{Spitzer, Gaia and the Potential of the Milky Way}

\author{Adrian~M.~Price-Whelan\altaffilmark{1,2} Kathryn V. Johnston\altaffilmark{1}}

% affiliations
\altaffiltext{1}{Department of Astronomy, Columbia University, 550 W 120th St., New York, NY 027, USA}
\altaffiltext{2}{NSF Graduate Fellow}

\begin{abstract}
.
\end{abstract}


\section{Introduction}
\label{intro.sec}
 The existence of vast halos of unseen {\it dark} matter surrounding each galaxy has long been proposed to explain the surprisingly large
motions of the {\it baryonic} matter (i.e. stars, gas and dust) that we can see \citep[e.g.,][]{zwicky37,rubin70}.
Dark-matter-only simulations of structure formation lead us to expect that these dark matter halos should have density distributions that are described by a universal radial profile \citep{navarro96} with a variety of triaxial shapes \citep{jing02}.
The inclusion of baryons in the simulations tends to soften the triaxiality of the dark matter in the inner regions of the halo \citep[e.g., as the disk forms,][]{bailin05} and
can potentially alter the radial profile through a combination of adiabatic contraction and energetic feedback \citep[see recent discussion by][]{pontzen12}.
Hence, measurements of the shape, orientation, radial profile, and extent of dark matter halos provides information about the formation of these vast structures, as well as the messy baryonic processes of gas inflow, star formation, and feedback that continue to shape them.

The Milky Way is the best candidate for such a detailed study of a dark matter halo. 
Our presence in the Galaxy means that we can resolve and use individual stars as tracers and hence build large samples that probe deeper into the halo than globular cluster
or planetary nebula studies of external galaxies \citep[e.g.][]{mendez01,cote03}. Our internal perspective allows us to measure the three-dimensional mass distribution around the Galaxy rather than a two-dimensional projection, such as in lensing studies \citep{bolton06}.
%Hence we can hope to reconstruct not only the density, but also the shape and orientation of the dark matter halo as function of radius well beyond the disk.
To date, thousands of blue horizontal branch stars selected from the Sloan Digital Sky Survey (SDSS) have been used to probe the Milky Way mass out to tens of
kpc \citep[SDSS, see][]{deason12a,kafle12}, and estimates with combined tracers extend to 150kpc \citep{deason12b}.
\citet{loebman12} demonstrates that even larger stellar samples from current (e.g., SDSS) and future  (e.g., Gaia and LSST) stellar surveys will also be sensitive to any
asphericity of the halo.

The caveat to working with tracer populations is the assumption that the tracer stars represent a random sampling of orbits drawn from a smooth distribution function and are well-mixed in orbital phase. However, the same surveys that provide large samples of tracers have also revealed the existence of large-scale spatial inhomogeneities in the form of giant stellar streams and clouds \citep{newberg02,majewski03,belokurov06}, clearly demonstrating that a significant fraction of the stellar halo is neither randomly sampled from a smooth distribution function nor is fully phase-mixed.
%The degree of substructure increases with Galactocentric radius \citep{bell08} in agreement with models where the stellar halo is formed purely from accretion events \citep{bullock05,cooper10,rashkov12}.
%Moreover, even in the smoother, inner halo which is likely to be phase-mixed, significant clumpings in velocity \citep{schlauffman10} and orbital properties \citep{helmi99} have been observed, which indicate a non-smooth component to the orbit distribution.
%By ``observing'' particles in a sample of simulated stellar halos, \citet{yencho06} showed that, if stellar halos are indeed entirely formed via accretion events, then the expected degree of non-randomness  could lead to systematic biases in mass estimates of order several tens of percent.
%This encouraging result suggests that, if our expectations for the typical number and size of accretion events contributing to the stellar halo are correct, then the number of orbits effectively explored is sufficiently large that the assumption of random, fully phase-mixed orbits is reasonable for the current samples.

% --- analogous to our exploitation of matter in disks moving on near-circular orbits.
A complimentary method to measure the mass distribution is to instead take full advantage of the {\it non}-random nature of the stellar distribution in the Galactic halo.
While we do not know the exact orbits of  stellar debris from satellite destruction -- though \citet{eyre09b} propose a method of full orbit reconstruction from incomplete kinematic data -- we do know something more about stars in a debris structure: the stars were once all part of the same object.
Methods that utilize this knowledge on dynamically cold systems (e.g, tidal streams) require orders of magnitude fewer tracers than a randomly sampled population, but achieve comparable accuracy on derived constraints.
In the simplest case, we might {\it assume} that debris stars are actually still on the same orbit as their progenitor system.
Then we can imagine that measuring the full-space velocities ${\bf v}$ at different points ${\bf x}$ in the structure (e.g. along a stream) would actually give us a direct
measure of differences in potential $\Phi$ (i.e. $\Phi({\bf x}_1)-\Phi({\bf x}_2)=(v_2^2-v_1^2)/2$).
In reality, we usually know at most four of the six phase-space co-ordinates for debris stars, often with significant error bars and potential measurements have been made by
fitting orbits to streams \citep[e.g.,][]{helmi04,johnston05,koposov10,law10}.
Moreover, our assumption of debris tracing a single orbit is actually incorrect \citep[see][for discussions of the orbit distribution in tidal debris]{johnston98,helmi99}, with
systematic changes in orbital energy along debris streams that can lead to systematic biases in our measurements of the Galactic potential \citep{eyre09a,varghese11}.
In particular, \citet{sanders13a} recently demonstrated that this bias is equally problematic for the very thinnest, coldest streams, whose observed properties may be indistinguishable from those of the parent orbit \citep[e.g. such as the globular cluster, GD1 --- see][]{koposov10}, as for the much more extended and hotter streams \citep[e.g. such as debris from the Sagittarius dwarf galaxy --- see][]{majewski03} where offsets from a single orbit are clearly apparent.

One way to address these systematic biases is to run fully self-consistent N-body simulations of satellite destruction in a variety of potentials (which naturally generate a
distribution of orbital properties in the debris and along tidal streams) with the aim of simultaneously constraining both the properties of the satellite and the Milky Way.
Many studies of the Sagittarius debris system (hereafter Sgr)  --- whose tidal streams entirely encircle the Milky Way --- have adopted this approach \citep[e.g.][]
{law05,fellhauer06} with the most recent work for the first time attempting to place constraints on the triaxiality and orientation of the Milky Way's matter distribution
\citep{law10}.
%Studies which first fit the Sgr stream to a single orbit \citep{johnston05,law09} and were subsequently followed up with full N-body simulations \citep{law05,law12} offer another demonstration of the limitation of the former approach as potential parameters are revised when moving to the N-body approach.

Other approaches have been proposed that avoid the cost of full N-body simulations yet recognize the limitations of single orbit integration.
For example, using the results of N-body simulations as a guide, simple corrections around a single orbit can be adopted to make model predictions that more closely
mimic expected debris behavior \citep[e.g.][]{johnston99b,varghese11}.

The promise of near-future data sets including full (or nearly full) phase-space information has also inspired discussions that move beyond fitting observables along a
stream.
\citet{binney08} and \citet{penarrubia12} demonstrate conceptually that the distribution of energy and entropy (respectively) in debris will be minimized only for a correct
assumption of the form of the Galactic potential.
\citet{sanders13b} examine the distribution of debris in action-angle co-ordinates, show stars that were once all members of the same object must lie along a single line in angle-frequency space and demonstrate how this constraint can be used as a potential measure.
%\citet{johnston99a} (hereafter JZSH) showed that backwards integration of the positions of stars and their parent satellite could be used to distinguish Milky Way mass models: only in the correct potential would the stars' and satellite's  path coincide within a Hubble time.The JZSH method was designed and tested for the proposed characteristics of the Space Interferometry Mission \citep{}, which was expected to be able to measure proper motions with $\mu$as/year accuracy for stars as faint as $V=20$. Distances to stream members would only be poorly known and were estimated using anticipated properties of tidal debris.

In this paper we re-examine and update  another approach to using tidal debris as a potential measure \citep[originially proposed by][]{johnston99a}  in the context of current and near-future observational capabilities, and illustrates its power with the example of Sgr.
In Section \ref{sec:context} we outline the observational prospects and Sgr properties that motivated this re-examination.
In Section \ref{sec:method} we present the updated potential measure and test it with synthetic observations of simulated Sgr debris.
In Section \ref{sec:discussion} we highlight the advantages and shortcomings of this method.
We conclude in Section \ref{sec:conclusion}.

% --------------------------------------------------------------------
% Above here is from KVJ. Below is APW.
%

\section{Context and motivation} \label{sec:context}
The method presented in Section \ref{sec:method} takes advantage of
three distinct developments in the observations of tidal streams: (i)
the demonstration of a technique for deriving distances to individual
RR Lyrae stars with 2\% accuracies (Section \ref{sec:spitzer}); (ii)
the prospect of proper motion measurements of the same stars with
$\sim$10~$\mu$as/yr precision (Section \ref{sec:gaia}); and (iii) the
tracing of debris associated with Sgr around the entire Galaxy
(Section \ref{sec:sgr})

\subsection{{\it Spitzer} and 2\% distance errors for RR Lyrae in the halo}
\label{sec:spitzer}

There is a long tradition for using RR Lyrae stars in the Galaxy to
study structure \citep[going back to first estimates of the distance
  to the Galactic center][]{shapley18}, substructure
\citep[e.g.][]{sesar10}, and distances to satellite galaxies
\citep[e.g.][]{clementini03}.  However, studies of RR Lyrae at optical
wavelengths are limited by both metallicity affects on the intrinsic
brightness of these stars and variable extinction along the line of
sight.  Moreover, systematic differences between instruments make it
difficult to tie observations across the sky to a common scale. While
RR Lyrae are among the best distance indicators available to us today,
the absolute uncertainty in individual estimates are typically
$\gtrsim$10\%.

At longer wavelengths, RR Lyrae promise tighter constraints on
distances.  The near-infrared period-luminosity (PL) relation for RR
Lyrae stars was first mapped by \citet{longmore86} using $K$-band
observations of stars in Galactic globular clusters and later shown to
be due to the changing bolometric correction for these stars as a
function of the observed wavelength \citep{catelan04}.
\citet{madore12} have recently shown that the intrinsic scatter of the
PL relation decreases with increasing wavelength. Both reddening
effects and metallicity dependencies also decrease significantly for
observations further into the IR.

These studies suggest that observations of RR Lyrae variables by
NASA's Spitzer mission could provide stellar distance indicators of
unprecedented accuracy and precision.  Using five stars with
trigonometric parallaxes measured by Hubble \citep{benedict11},
\citet{madore12} measured the dispersion in the mid-IR PL relation at
Spitzer wavelengths to be $\sim$0.03~mag. This implies that it is
possible to determine individual distances that are good to better
than $2\%$ for RR Lyrae stars out to $\sim$60~kpc (the limit of
Spitzer's ability to detect and measure RR Lyrae variables for modest
integration times). Distance measurements of Blue Horizontal Branch
and Blue Straggler stars at these distances in the halo typically
achieve $\sim$25\% uncertainties \citep[e.g.,][]{deason12b}.


\subsection{Gaia and the age of astrometry}
\label{sec:gaia}
The Gaia satellite \citep{gaia01}, planned for launch in late 2013, is
an astrometric mission which aims to measure the positions of billions
of stars with 10-100~$\mu$as accuracies. With over 50 observations of
the entire sky during the five-year lifetime of the mission, Gaia will
revolutionize the field of Galactic Astronomy by providing sufficient
proper motion accuracy to create full six-dimensional phase-space maps
of the Galaxy with $<$10\% distance errors for heliocentric distances
up to $\sim$6~kpc for RR Lyrae stars.

Figure~\ref{fig:gaia_errors} shows the Gaia end-of-mission distance
and tangential velocity error estimates for RR Lyrae. The velocity
uncertainty depends on the magnitude of the source, which depends on
the metallicity: each line in Figure~\ref{fig:gaia_errors} is computed
by Monte Carlo sampling from the empirical metallicity distribution of
the Galactic halo from \cite{ivezic08} (centered around [Fe/H] $\sim$
-1.5). Within 2~kpc, Gaia will measure distances to these stars with
better than 2\% accuracy -- RR Lyrae in this volume will be used to
test and calibrate the Spitzer PL relation described above. Beyond the
2~kpc threshold, the mid-IR PL relation for RR Lyrae will provide more
accurate distance measurements.

\begin{figure}[h]
\begin{center}
\includegraphics[width=1.\textwidth]{/Users/adrian/projects/streams/plots/paper1/gaia.pdf}
\caption{ todo }\label{fig:gaia_errors}
\end{center}
\end{figure}

The anticipated proper motion errors for RR Lyrae correspond to
tangential velocity errors of order $\sim$1~km/s at 30~kpc, and of
order $\sim$10~km/s out to several tens of
kiloparsecs.\footnote{\url{http://www.rssd.esa.int/index.php?page=Science_Performance&project=GAIA}}
The combination of Spitzer and Gaia data will extend the ``horizon''
of where full six-dimensional phase-space maps of the Galaxy are
possible from $<$10~kpc to 80~kpc. This enormous increase in volume
will greatly refine data on existing debris systems such as the
Sagittarius stream.

\subsection{The Sagittarius debris system}
\label{sec:sgr}
The Sagittarius Dwarf Galaxy (hereafter Sgr) was discovered
serendipitously during a radial velocity survey of the Galactic bulge
\citep{ibata94}. While the number of known Milky Way satellites has
more than doubled recently \citep[e.g.,][]{mcconnachie12}, in many
ways Sgr remains the most studied and most intriguing: it is the
closest satellite, among the largest (only the LMC and SMC are more
luminous), and has the most prominent tidal tails.

The first maps of the Sgr core \citep{ibata94, ibata95} revealed its
elongated morphology and suggested tidal deformation, as would be
anticipated from its proximity to the Galaxy. Subsequent N-body models
confirmed this interpretation \citep{velazquez95} and showed that
debris from Sgr should form coherent streams of stars that encircle
the Galaxy \citep{johnston95}. Signatures of these extensive stellar
streams have since been discovered in many studies and have been
mapped across the sky in carbon stars \citep{totten98}, M giants
selected from the Two Micron All Sky Survey \citep{majewski03}, main
sequence turnoff stars from the Sloan Digital Sky Survey
\citep{belokurov06}, and RR Lyrae in the Catalina Sky Survey
\citep{drake13}. More recently, apogalactic debris has been traced to
large distances using a combination of blue horizontal branch,
main-sequence turn-off, and red giant stars from the SDSS
\citep{belokurov13}.

The distance to the stream has been estimated from photometric surveys
\citep[e.g.,][]{martinezdelgado04} and gradients in radial
velocities and metallicity measured with follow-up spectroscopy
\citep[e.g.,][]{majewski04, vivas05, bellazzini06, chou07,
  chou10, keller10, carlin12}. The multitude of Sgr stream data
has inspired a rich set of interpretations and models of the debris
\citep[e.g.,][]{johnston99b, helmi04, law05, fellhauer06,
  law10}.

Most recently, \citet[][hereafter LM10]{law10} have combined all
current data on the Sgr debris to assess the triaxiality and
orientation of the outer Galaxy, the first time that such a
reconstruction of the 3-dimensional mass distribution of a dark matter
halo has been feasible. Figure \ref{fig:lm10} shows particle positions
(in Galactocentric coordinates) from the final time-step of an N-body
simulation of dwarf satellite disruption along the expected Sgr orbit
in the best-fitting Milky Way halo model from LM10. The simulation was
run in a three-component potential, with a triaxial, logarithmic halo
model of the form
\begin{equation}
  \Phi_{halo} = v_{halo}^2 \ln(C_1 x^2 + C_2 y^2 + C_3 xy + (z/q_z)^2 + R_c^2)
\end{equation}
where $C_1$, $C_2$, and $C_3$ are combinations of the $x$ and $y$ axis
ratios ($q_1$, $q_2$) and orientation of the halo with respect to the
baryonic disk ($\phi$):
\begin{align}
  C_1 &= \frac{\cos^2\phi}{q_1^2} + \frac{\sin^2\phi}{q_2^2}\\
  C_2 &= \frac{\sin^2\phi}{q_1^2} + \frac{\cos^2\phi}{q_2^2}\\
  C_3 &= 2\sin\phi\cos\phi \left(q_1^{-2} - q_2^{-2}\right).
\end{align}

LM10 demonstrate how a comparison of such simulations with the
observed debris enables a detailed reconstruction of Sgr's history:
its mass and stellar populations, orbit, rate of destruction, and even
original distribution in populations. The phase-space distribution of
the debris also offers a unique probe of the depth, shape, extent, and
orientation of the Milky Way's dark matter halo that cannot be matched
using other techniques on our own or other galaxies
\citep[e.g.,][]{ibata01}. The LM10 constraints on Sgr's mass and orbit
have invigorated discussions of Sgr in a more cosmological context,
the mass and extent of the original dark matter halo that hosted it,
as well as its effect on the Milky Way itself
\citep{bailin03,purcell11,micheldansac11,gomez12}.

Combined Spitzer and Gaia measurements of distances and proper motions
of RR Lyraes in the Sgr debris will open up new avenues of
interpretation. A 2\% distance error at $\sim$30~kpc
($\epsilon_{d}=0.6~{\rm kpc}$) is smaller than the predicted thickness
of the stream, ranging from $\sim$2-$10$~kpc. Combined with Gaia
tangential velocity measurement accuracies ($\epsilon_{v}\sim2$~km/s
at 30~kpc) less than the intrinsic dispersion for much of the stream
\citep[$\sigma_v\sim10$~km/s;][]{majewski04}, and ground-based radial
velocities, the measured phase-space position of each individual RR
Lyrae becomes a powerful probe of the potential. The next section
outlines a new method to take advantage of this information.

\section{Description and test of our algorithm}
\label{sec:method}
With access to full phase-space information for stars in a tidal
stream, it is possible to use the individual stars as potential
measures by exploting the fact that the stars must have come from the
same progenitor. If the present-day phase-space position of the
progenitor system is also known, this places an intuitive constraint
on the potential of the Milky Way: if the orbits of the stars and the
orbit of the progenitor are integrated \emph{backwards} in time
(imagine watching satellite destruction in ``rewind"), the stars
should recombine with the progenitor. If the potential is incorrect,
the orbits of the stars will diverge from that of the progenitor and
thus will not be recaptured by the satellite system
(Figure~\ref{fig:ps_distance}).

This approach was originally proposed by \citep{johnston99a} and was
designed and tested for the proposed characteristics of the Space
Interferometry Mission \citep{unwin08}, which was expected to be able
to measure proper motions with $\mu$as/year accuracy for stars as
faint as $V=20$. Distances to stream members would only be poorly
known and were estimated in the method by using anticipated properties
of tidal debris. Below we present an updated version of the algorithm:
the promise of 2\% distances to RR Lyrae stars (see Section
\ref{sec:spitzer}) enables a direct measurement (rather than
approximate estimate) of the position of a star within its debris
structure. The test statistic that quantifies how well stars recombine
with the satellite has been rigorously redefined.

\begin{figure}[h]
\begin{center}
\includegraphics[width=1.\textwidth]{/Users/adrian/projects/streams/plots/paper1/ps_distance.pdf}
\caption{ todo }\label{fig:ps_distance}
\end{center}
\end{figure}

\subsection{The algorithm: Rewinder}
Quantifying this method requires a sample of stars with known full
space kinematics $(\bs{x}_{i}, \bs{v}_{i})|_{t=0}$ (e.g., measurements
of all position and velocity components for these stars \emph{today}
at $t=0$), the orbital parameters for the progenitor system
$(\bs{x}_p, \bs{v}_p)|_{t=0}$, and a functional form for the
potential, $\Phi({\boldsymbol\theta})$. For a given set of potential
parameters, $\boldsymbol\theta$, the orbits of the stars and
progenitor are integrated backwards for several Gigayears (determined
from an estimate of the interaction time of the system). The stars and
progenitor system are treated as test particles, so that the debris
stars do not become physically re-bound to the satellite. Instead, at
each timestep $t_j$, for each particle $i$, a set of normalized,
relative phase-space coordinates are computed
\begin{equation}
  \bs{q}_{i} = \frac{\bs{x}_{i} -
    \bs{x}_{p}}{R_{\rm tide}}\,\,\,\,,\,\,\,\,\bs{p}_{i} = \frac{\bs{v}_{i} -
    \bs{v}_{p}}{v_{\rm esc}}
\end{equation}
where $(\bs{x},\bs{v})_{i}$ and $(\bs{x},\bs{v})_{p}$ are the
phase-space coordinates for the particles and progenitor,
respectively. These definitions require an estimate of the mass of the
satellite, $m_{sat}$, which, combined with the orbital radius of the
satellite, $R$, and the computed enclosed mass of the potential within
$R$, $M_{enc}$, sets the instantaneous tidal radius and escape
velocity,
\begin{equation}
  R_{\rm tide}=R\Big(\frac{m_{\rm sat}}{3M_{\rm enc}}\Big)^{1/3}\,\,\,\,,\,\,\,\,
  v_{\rm esc}=\sqrt{\frac{2Gm_{\rm sat}}{R_{\rm tide}}}.
\end{equation}
Qualitatively, when the distance in this normalized six-dimensional
space, $D_{{\rm ps},i}=\sqrt{|\bs{q}_{i}|^2+|\bs{p}_{i}|^2} <
\sqrt{2}$, the star is likely recaptured by the
satellite. \citet{johnston99a} (hereafter JZSH) imposed a similar
condition as a hard boundary and used the number of recaptured
particles in a given backwards-integration as a cost function for
optimization. What follows is a description of an updated procedure
with a more natural, statistically-motivated choice for an objective
function.

%As a measure of how many stars have recombined, we treat the minimum phase-space vectors (for each star over the individual, integrated orbits) as samples from an underlying six-dimensional distribution.

For each star, $i$,
the phase-space distance, $D_{\rm ps}$, is computed at each timestep
$t_{j}$, and the vector with the minimum phase-space distance is stored
\begin{align}
  t^*_{i} &= \argmin_{t} D_{{\rm ps},i}\\
  \bs{A}_{i} &= (\bs{q}_{i}(t^*_{i}),\bs{p}_{i}(t^*_{i})).
\end{align}
Thus, the matrix $\bs{A}_{ik}$ contains these minimum phase-space
distance vectors for each star, where $k\in[1,6]$. Intuitively, the
variance of the distribution of minimum phase-space vectors will be
larger for orbits integrated in an incorrect potential relative to the
distribution computed from the `true' orbital history of the stars: in
an incorrect potential, small differences in the orbits of the stars
relative to the orbit of the progenitor cause them to spread out in
this relative, normalized phase space. Thus, the \emph{generalized
  variance} of the distribution -- computed for a given set of
potential parameters, $\bs{\theta}$ -- is a natural choice for the
scalar objective function used in constraining the potential of the
Milky Way
\begin{align}
  \Sigma_n &= \mathrm{Cov}( \bs{A}_{ik}) \\
  f(\bs{\theta}) &= \ln \det \Sigma.
\end{align}

[Still working on this bit...need to figure out best way to express
  what I'm trying to say.] Though this is not explicitly a
\emph{likelihood}, optimizing over this statistic is a very simple
approximation to performing a maximum-likelihood
optimization. Assuming the samples in $\bs{A}_{ik}$ are approximately
normal-distributed, the variance is closely related to the Shannon
entropy \citep{shannon1948}, $H$,
\begin{equation}
  H \propto \log \sigma .
\end{equation}
Minimizing the entropy is then equivalent to minimizing the loss of
\emph{information} for a given choice of parameters, $\bs{\theta}$,
given the ``data,'' $\bs{A}_{ik}$, and is thus related to
maximum-likelihood estimation \citep[see][for more
  detail]{mackay2002}.

\subsection{Application to Simulated Data} \label{sec:results}
The LM10 simulation data (see Section \ref{sec:sgr}) is a perfect
test-bed for evaluating the effectiveness of this method. We start by
extracting both particle data and the satellite orbital parameters
from the present-day snapshot of the simulation
data.\footnote{\url{www.astro.virginia.edu/~srm4n/Sgr/data.html}} We
then `observe' a sample of 100 stars from the first leading and
trailing wraps of the stream: the radial velocity errors are drawn
from a Gaussian $\epsilon_{rv} \sim \mathcal{N}(\mu=0,\sigma=10~{\rm
  km/s})$, the tangential velocity errors depend on the distance to
the star and are computed from the expected Gaia error curve (shown in
Figure~\ref{fig:gaia_errors}), and the distance errors on a star of
distance $D$ are $\epsilon_{dist} \sim \mathcal{N}(0,0.02\times
D)$. Figure~\ref{fig:variance_proj} shows two-dimensional projections
of the six-dimensional minimum phase-space distance distribution for
the observed particles in the correct and an incorrect potential: it
is clear that the variance of the distribution increases in the wrong
potential. The generalized variance defines a convex hypersurface over
which we optimize four of the six logarithmic potential parameters:
$v_{circ}$, $\phi$, $q_1$, and $q_z$. We fix $q_2$ and $R_c$, which
are degenerate with combinations of the other
parameters. Figure~\ref{fig:objective} shows one-dimensional slices of
the objective function produced by varying each of the potential
parameters by $\pm10\%$ around the true values and holding all others
fixed.

\begin{figure}[h]
\begin{center}
\includegraphics[width=1.\textwidth]{/Users/adrian/projects/streams/plots/paper1/variance_projections.pdf}
\caption{ todo }\label{fig:variance_proj}
\end{center}
\end{figure}

\begin{figure}[h]
\begin{center}
\includegraphics[width=1.\textwidth]{/Users/adrian/projects/streams/plots/paper1/objective_function.pdf}
\caption{ todo }\label{fig:objective}
\end{center}
\end{figure}

In anticipation of extending the above method to include a true
likelihood function, we use a parallelized Markov Chain Monte Carlo
(MCMC) algorithm \citep{foremanmackey2013} to sample from our
objective function.\footnote{Though MCMC is typically not an efficient
  optimization tool, in this case objective function is both noisy and
  expensive to compute. The stochasticity and easy parallelization of
  the algorithm mean it outperforms other optimizers on this problem.}
We use the median value of the (burnt-in) sample distribution as a
point estimate for the potential parameters. To assess the uncertainty
in the derived halo parameters, we bootstrap resample our 100 stars
100 times and estimate the potential parameters with each
resampling. Figure~\ref{fig:bootstrap} shows the recovered parameters
for each bootstrapped sample and demonstrates the power of the
combined Spitzer and Gaia data: a moderately sized sample of RR Lyrae
alone places quite strong constraints on the shape and mass of the
Galaxy's dark matter halo.

% APW Note: The optimization, in a way, is getting a bunch of orbits (curves) to intersect another curve (the progenitor orbit) at some points (over time).

\section{Discussion and future prospects}
\label{sec:discussion}

Many orbit fitting methods for measuring the Galactic potential assume
that the stars in a given stream are on the same
orbit. \cite{sanders13a} show that this assumption can lead to nearly
100\% errors in the estimated potential parameters, even without
realistic observational errors on the kinematic data. The
\emph{Rewinder} is free of this assumption and instead \emph{relies}
on the fact that each star is on a different orbit, and is thus an
independent potentiometer. Qualitatively, the best potential is the
one that gets the most of these orbits to intersect the progenitor
over the history of interaction with the system.

The strengths of this method stem from its simplicity: it requires
only a rough estimate of the satellite mass $m_{\rm sat}$ combined
with backwards integration of orbits. There are no assumptions made
about, or N-body modeling of, the internal distribution of satellite
stars. Thus, the \emph{Rewinder} is applicable to any debris that is
known to come from a single object and not restricted to the coldest
tidal streams. In principle, the method could also be applied to the
vast debris {\it clouds} that have been discovered over large areas of
the sky \citep[e.g. the Triangulum Andromeda and Hercules-Aquila
  clouds]{rochapinto04,belokurov06}, or even stars that have only
associations in orbital properties and do not form a coherent spatial
structure \citep[such as the angular momentum groupings in local
  giants found by][]{helmi99}. The method trivially extends to
combining constraints from multiple debris systems at once by simply
integrating all debris from several satellites simultaneously, with
$D_{\rm ps}$ defined appropriately for each star.

It is also important to characterize the the limitations of this
method. Firstly, the measurement errors for RR Lyraes associated with
the very coldest streams \citep[e.g., the globular clusters Pal5 and
  GD1;][]{odenkirchen02,koposov10} will likely be too large to resolve
the minute differences in orbital properties between the debris and
satellite to be detected. Second, the current version of the algorithm
relies on accurate knowledge of the current position and velocity of
the parent satellite, which may not be available in some cases
\citep[for example, the Orphan Stream;][]{belokurov07}. Lastly, the
present prescription treats the progenitor as a point-mass during
orbit integration. Second-order effects in orbit evolution (e.g.,
dynamical friction) are currently ignored, leading to minor biases in
the estimated potential parameters (for example, XXX in
Figure~\ref{fig:bootstrap}).

[KVJ: Ok, there is a question as to whether we even want to say something like the below...?]

The objective function presented above is an approximation to a more
general likelihood function, described in depth in forthcoming work
(Price-Whelan \& Johnston in prep.). The results shown in section
\ref{sec:results} are meant to illustrate the strength and concept
behind the method, but [...] [I want to say one more sentence about
  the upcoming work, and possibly how it will solve some of the
  limitations presented above.]

%With a justified probabilistic framework, we will also be able to provide iven sufficient data, position and motion of the original host could be solved for as additional free parameters, but the effectiveness of this approach has yet to be investigated.

\section{Conclusions}
\label{sec:conclusion}

This paper presents an algorithm for measuring the Galactic potential
that anticipates combined data from the Spitzer and Gaia satellite
missions that promise precise, full phase-space measurements of RR
Lyrae stars in the halo of our Galaxy. When applied to a sample of 100
stars (with realistic observational errors) sampled from the
\cite{law10} N-body simulation of the destruction of the Sgr dwarf
satellite, the method recovers the shape and orientation of the dark
matter potential XXX. This success provides strong motivation for 1)
further theoretical work to develop a robust generative model (and therefore likelihood) that utilizes the concepts demonstrated by the \emph{Rewinder}, 2) an investigation of the power of
using multiple, complimentary debris structures (e.g., with different
morphologies and orientations) in building a comprehensive picture of
the Galactic mass distribution, and 3) a Spitzer survey of RR Lyrae
stars in debris structures around the Milky Way to get precise
distances to combine with near-future Gaia velocity data. We conclude by noting that this method is complimentary to the orbit fitting algorithm presented in \citet{sanders13b} by favoring warmer, more massive debris structures like Sgr. By developing and comparing independent methods for measuring the potential of the Milky Way, we will begin to develop a deeper understanding of the vast structures of dark matter that permeate the universe. [Yea, the last sentence I wrote in the late night and am probably delirious...]

\acknowledgments
We thank ...

APW is supported by a National Science Foundation Graduate Research
Fellowship under Grant No.\ 11-44155.

\bibliography{refs}
\bibliographystyle{apj}

\end{document}
