\documentclass[11pt,letterpaper,dvips]{article}
 
 
\pretolerance=10000
\textwidth=7.0in
\textheight=10.0in
\voffset = -0.3in
\topmargin=0.0in
\headheight=0.00in
\hoffset = -0.3in
\headsep=0.00in
\oddsidemargin=0in
\evensidemargin=0in
\parindent=2em
\parskip=1.5ex
 

\newcommand{\nnn}{ccd\#\#\#}
 
\special{papersize=8.5in,11in}

\renewcommand{\theenumi}{\Roman{enumi}}

\begin{document}
%\pagestyle{empty}
 

\begin{center}
{{\Huge \bf LDSS3 Long-Slit Reductions}}\\
{{\Huge \bf [v1.0; March 2005]}}
\end{center}


\begin{enumerate}
{\Large \bf \item LDSS3 General Info} 

This code reduces data from Magellan/LDSS3 and is written for
single-slit observations only.  Raw LDSS3 data consists of four fits
files corresponding to each of the four CCD chip amplifiers. Raw data
files are named ccd***c[1-4].fits.  The CCD chip layout is as follows:
c1 = lower left CCD quadrant, c2 = lower right, c3 = upper right, c4 =
upper left.  The code stitches the four quadrants together and
operates on combined frames for all reduction operations after bias
subtraction.  First time users should run the invidual steps listed
below, however the data can be reduced in a single step by running the
wrapper script xxxx.  Example fits files of intermediate processing
steps can be found in LDSS3/examples.  Comments and questions should
be sent to M. Geha at mgeha@ociw.edu.


{\Large  \bf \item LDSS3 Suggested Long-Slit Calibrations}

   \begin{itemize}
	\item 10 bias (zero exposure) frames.
	\item 1 Arc at position of science frames.
	\item 2 QTZ Lamps at position of science frames (especially if working
	  redward of $6700\mbox{\AA}$).
	\item Bright imaging flats to determined gain factor.
	\item All calibrations should be taken with the same readout
	  speed (slow/fast/turbo) as the science data.

   \end{itemize}   


{\Large \bf \item Required Software } 
  \begin{itemize}
     \item The LDSS3 code requires a copy of IDL version 5.4 or higher
      and relies on several IDL software libraries.  Download and
      installation information are available at the websites listed below.
	\item  IDLUTILS/IDLSPEC2d mantained by SDSS available at\\
	http://spectro.princeton.edu/idlspec2d\_install.html

	\item XIDL package mantained by JXP available at\\
	http://www.ucolick.org/$\sim$xavier/IDL/index.html
   \end{itemize}

\vskip 0.5cm
{\Large \bf \item Initial Setup (Repeat for each night) }

  \begin{itemize}

    \item Create a new directory for the night's data (eg. obs030205/).

    \item In the night directory, create a rawdata/ directory and put all
    the raw data in it.  The code assumes data is gziped and is named
    ccd***.fits.gz.

    \item Create a setcrds file in directory above rawdata/
      (eg. setcrds\_obs030205.pro).  This file sets which exposures
      are flats/arcs/science frames.  An initial guess at this
      information is generated by ldss\_setup.pro (see below) from
      information in the headers.  The setcrds routine explicitly sets
      which exposures are reduced and which calibrations are
      associated with a given science exposure.  This file is called
      by ldss\_setup.

    \item An example setcrds file is available in
      {\tt /examples/setcrds\_obs030205.pro}.  For a given science object,
      the following lines are required:\\

        \begin{itemize}
	\item Note the IDL routine {\bf fnd\_indx} converts
	the frame number of an image to the index in the structure. 


        strct[fnd\_indx(strct,59):fnd\_indx(strct,67)].objname   = 'galaxy\_name'    ; name of galaxy\\
	strct[fnd\_indx(strct,59):fnd\_indx(strct,67)].objnum  = 1   ; unique ID for exposures associated with science target\\
	strct[fnd\_indx(strct,59):fnd\_indx(strct,67)].flg\_anly= 1   ; analyze frames\\
	strct[fnd\_indx(strct,63):fnd\_indx(strct,64)].type    = 'OBJ'~~~~; science frames\\
	strct[fnd\_indx(strct,65)].type = 'ARC'~~~~~~~~~~~~~~~~~~~~~~~~~~~; arc lamp\\
	strct[fnd\_indx(strct,66):fnd\_indx(strct,67)].type    = 'FLT'~~~~; flat field\\
	
	\item If strct[*].flg\_anly = 0 an exposure is not analyzed (eg. a bad flat field or alignment
	  exposures.)

        \end{itemize}
	

	 \item The routine ldss\_procnight runs all of the routine
	   below for a given night.  I suggest first running the
	   routines individually before using this routine.

     \end{itemize}




{\Large \bf \item Calibrations }

{\Large \bf Create LDSS3 Structure }
  
   \begin{itemize}
    \item Launch idl in the directory above rawdata/


    \item {\bf ldss\_setup} :: Create the ldss structure.  This routine
      organizes the night's data and is used in all subsequent processing steps.
      This routine creates several things:
         \begin{enumerate}
	   \item  Empty directories (Arcs/ Flats/ Final/) where intermediate
	     processing steps will be stored.
	   \item  An IDl structure in memory with whatever name you chose.
	   \item  An IDl structure written to a .fits file.
         \end{enumerate}
    \item This routine first creates the directories, then searches for data
      files in rawdata/.  Header information is used to create the ldss
      structure.  This information is then modified by calling the 
      setcrds file created above.
      
    \item See Section xx for the tags set in the ldss structure.

    \item This routine outputs a .fits copy of the ldss structure
      named ldss\_ + night + .fits (eg. ldss_obs030205.fits).  If you
      re-start IDL, you will need to read this structure into memory:
      IDL$>$ strct = mrdfits('ldss_obs030205.fits',1)

         \quad Example: IDL$> \;$ {\bf ldss\_setup}, 'obs030205', strct \\  
   \end{itemize}

{\Large \bf Create Bias Frame }

	\begin{itemize}
	  \item {\bf ldss\_mkbias} :: Make bias frame for each of 4
	    LDSS3 chips. Median combines all bias frames as defined in
	    the structure with strct.tpye = 'ZRO'

	  \item There is two-dimensional structure in the bias frame
	    that cannot be removed via overscan subtraction alone.

	  \item  This routine outputs four images in the Bias/biasc* directory.  
	   
	  \item To check the Bias frame : {\bf xatv}, 'Bias/Biasc*.fits'

         \quad Example: IDL$> \;$ {\bf ldss\_mkbias}, strct, /plot \\

	\end{itemize}


{\Large \bf Set Gain Factor }
     \begin{itemize}
       \item {\bf ldss\_gaincalc} :: Determine the gain factors
	 relaticve to chip 1.
       \item As of March 2005, the relative gain factors for the four LDSS3
	 was not constant between runs and depended on the readout
	 speed (eg. slow, fast, turbo).  As the instrument is used
	 more these values could probably be hardwired into the code.
	 For now, we need to determine this empirically and add them
	 to the ldss structure.

       \item In setcrds file, define a few bright imaging flat fields to determine
	 gain factor (must be same readout speed as science exposures).  These will
	 have stct.type = 'GFT'. 

         \quad Example: IDL$> \;$ {\bf ldss\_gaincalc}, 'obs030205', strct, /plot \\

     \end{itemize}

{\Large \bf Create Rough Arc Solution }

	\begin{itemize}
	  \item {\bf ldss\_idarc} :: Create a rough wavelength
           solution based on a stacked arc image.  This solution will
           serve a starting guess for reducing arc frames associated with
           the science exposures.

	  \item This routine requires the file ldss\_henear.dat
	    included in the ldss tarfile.  The user is required to
	    identify the wavelength and corresponding pixel value for
	    3-4 arc lines as input to this routine

	  \item Writes a rough wavelength solution to Arc/arcsol.fits

         \quad Example: IDL$> \,$  hand_lambda = [4471.47, 5015.675, 5944.834, 6402.246]\\ 
	                IDL$> \,$  hand_pix    = [342,     1172,     2507,      3146]\\
			IDL$> \,$ {\bf ldss\_idarc}, strct, hand\_lambda= hand\_lambda, hand\_pix= hand\_pix, /plot \\

	\end{itemize}



{\Large \bf \item Process Science Frames }

	\begin{itemize}
	\item {\bf ldss\_procobj} :: This routine processes raw
	  science frames.  Science frames are bias subtracted,
	  overscan subtracted, flat fielded, flaged for cosmic rays
	  and sky subtracted.  At the moment, it is not possible to
	  turn off any of these steps, but this could be modified.

         \quad Example: IDL$> \;$ {\bf ldss\_procobj}, strct, objnum, imred \\

	 \item This routine calls the following subroutines:

	\item {\bf ldss\_stitch4} :: Bias subtract and stitch the four LDSS3
	  CCD chips together.
	    \begin{itemize}
	      \item This code can also be run as a stand alone routine
                 to stitch a single LDSS3 frame together.  Data
                 reduction proceeds on stitched frames
		 \quad Example: IDL$> \;$ {\bf ldss\_stitch4}, im, inroot='rawdata/ccd0030'
	    \end{itemize}

	  \begin{enumerate}
	  \item {\bf ldss\_mkflat} :: Process flat field images
	  associated with given science target. 
	    \begin{itemize}
	      \item This routine processes and combines the flat field
		images associated with the science exposure
		(preferably taken at the same sky position).  The
		combined flat field to normalized by fitting a
		polynomial (order = 7) to the spatially collapsed flat.
	      \item  Output .fits file Flats/flat\_[objnum].fits
	    \end{itemize}

	\item Create inverse varience image ivar = im * gain + RN$^2$

	\item Divide by normalized flat field and trim image (default
	trim section = [*,1300:2500])

	\item {\bf ldss\_mkmask} :: Mask regions of zero flux (intended
	  for the bridge regions of the LDSS3 longslits).  


	\item {\bf ldss\_crflag} :: Flag cosmic rays, outputs indices
	  of all cr pixels which are to be set to zero (these pixels not used
	  in subsequent processing steps).
	    \begin{itemize}
	      \item Cosmic rays are identified as being sharper than
		the image PSF.  Based on the IDLUTILS routine {\bf
		reject\_cr}.
	    \end{itemize}

	\item {\bf ldss\_fixcol} :: Interpolate over bad columns are
	defined in the file ldss\_badcol.dat.

	\item{\bf ldss\_arcsol} :: Create a 2D wavelength array.
	    \begin{itemize}
	      \item Processes arc lamp image associated with science
                frame (preferably taken at same sky position).  For
                each row, solve for wavelength solution.  Initial fit
                taken from results of ldss\_idarc (see above).
	      \item For each row of arc
	      \item Writes to Arc/arc\_[objnum].fits
	    \end{itemize}

	\item {\bf ldss\_skysub} :: Sky subtract single image via
	bspline fitting.
	    \begin{itemize}
	      \item Fits a bspline to the collapsed sky region (sky can
               be defined in ldss\_procobj, default sky = [*,0:199] and
               [*,900:*]).  Based on the IDLUTILS routine {\bf bspline\_iterfit}.
	    \end{itemize}

	\item Write reduced image to Final/f\_ccd**.fits.  The final images
	  are .fits files with three image planes/axes.  These are
	        img  = mrdfits('Final/f\_ccd*.fits', 1)~~~ reduced image  
	        ivar = mrdfits('Final/f\_ccd*.fits', 2)~~~ inverse varience  
	        wave = mrdfits('Final/f\_ccd*.fits', 3)~~~ wavelength at each pixel  


        \end{enumerate}
	\end{itemize}

{\Large \bf \item Combine and Extract Science Spectra }

    \begin{itemize}
	\item {\bf ldss\_combspec} :: This routine combines multiple object
	structures (primarily their spectra) into one Final structure. 
	For objects with multiple
	spectra, the spectra are matched to the same flux and summed.  At
	the moment this routine only handles two spectra at a time.  \\
	\item Output :: A structure containg all of the fluxed, coadded spectra
         \quad Example: IDL$> \;$ {\bf ldss\_combspec}, strct, objnum, imcomb\\
  \end{itemize}



\clearpage
{\Large \bf \item LDSS3 Reduction Structure }
	\begin{center}
	{\Large {\bf ldssstr}}
	\end{center}

	{\small
	\begin{tabular}{lcl}
	  \hline
	  Tag & Type & Comment \\
	  \hline
         ccdframe   & ' '   & CCD FRAME name (in /rawdata) \\
         ccdnum     & 0     & CCD FRAME number\\
         flg\_anly & 0      &  Analysis flag 0=Don't Analyse \\
         type      & ' '    & ObjTyp : OBJ, STD, DRK, ZRO, FLT, ARC \\
	 objname   & ' '    & Science object name \\
	 objnum    & ' '    & Science object number (unique for a objects in a given night)\\
         exp       & 0.d    & Exposure time \\
         dateobs   & 0.0d   & Date of Obs \\  
         UT        & ' '    & UT \\
         RA        & 0.0    & RA \\
         DEC       & 0.0    & DEC \\
         SRA       & ' '    & RA \\
         SDEC      & ' '    & DEC \\
         Equinox   & 0.     & EQUINOX \\
	 airmass   & 0.0    & airmass\\
	 rotang    & 0.0    & Rotation angle\\
         grism     & ' '    & Grism : VPHB, VPHR, MEDR \\
         slitwidth & 0      & Aperture position \\
         gain     & fltarr(4) & Gain for each chip\\
         readno   & fltarr(4) & Read Noise for each chip \\
  
	  \hline
	\end{tabular}
	} 
	
\end{enumerate}


\hline
This software was written and is mantained by M.Geha (OCIW).  Please
send questions, software bugs or comments to mgeha@ociw.edu.

\end{document}
	

